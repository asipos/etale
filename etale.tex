\documentclass[13pt]{book}
\usepackage[romanian]{babel}
\usepackage{euler}
\usepackage{xltxtra}
\usepackage{color}
\usepackage{indentfirst}
\usepackage{mathtools}
\usepackage{amsmath}
\usepackage{amsfonts}
\usepackage{amsthm}
\usepackage{amssymb}
\usepackage{listings}
\usepackage[margin=1.1in]{geometry}
\usepackage[linktoc=none]{hyperref}
\usepackage{url}
\usepackage[nottoc]{tocbibind}
\usepackage[raggedright]{titlesec}
\setmainfont{URW Palladio L}
\setcounter{secnumdepth}{-1}
\setcounter{chapter}{0}
\lstset{
language=Haskell,
basicstyle=\ttfamily\fontsize{2.33mm}{4mm}\selectfont,
numbers=left,
numberstyle=\ttfamily\fontsize{2.6mm}{3mm}\selectfont,
stepnumber=3,
numbersep=10pt,
frame=single
}

\begin{document}

\newtheorem{teo}{\bf Teorema}[chapter]
\newtheorem{cor}[teo]{\bf Corolarul}
\newtheorem{lema}[teo]{\bf Lema}
\newtheorem{prob}[teo]{\bf Problema}
\newtheorem{conj}[teo]{\bf Conjectura}
\newtheorem{propr}[teo]{\bf Proprietatea}
\newtheorem{prop}[teo]{\bf Propoziția}
\newtheorem{alg}[teo]{\bf Algoritmul}
\theoremstyle{remark}
\newtheorem{obs}[teo]{\bf Observația}
\newtheorem{obss}[teo]{\bf Observațiile}
\newtheorem{ex}[teo]{\bf Exemplul}
\newtheorem{exs}[teo]{\bf Exemplele}
\theoremstyle{definition}
\newtheorem{defi}[teo]{\bf Definiția}

\pagenumbering{roman}
\setcounter{page}{1}
\fontsize{3.9mm}{5mm}\selectfont
\pagestyle{empty}
\begin{center}

\LARGE UNIVERSITATEA DIN BUCUREȘTI\\[0.5cm]
\LARGE Facultatea de Matematică și Informatică\\[3cm]

\Large Disertație masterală\\[3.5cm] 
 
\rule{8cm}{0.5mm}\\[0.8cm]
{ \huge \bfseries Étale stuff}\\[0.2cm]
\rule{8cm}{0.5mm}\\[2.5cm]

\begin{minipage}{0.4\textwidth}
\begin{flushleft} \large
\emph{Autor:}\\
Andrei Sipoș
\end{flushleft}
\end{minipage}
\begin{minipage}{0.5\textwidth}
\begin{flushright} \large
\emph{Profesor coordonator:} \\
Lect. dr. Victor Vuletescu
\end{flushright}
\end{minipage}

\vfill

{\large București, 2014}

\end{center}
\newpage
\phantom{X}
\newpage

\pagestyle{headings}
\tableofcontents
\newpage

\setlength{\parskip}{1.5ex plus 0.5ex minus 0.2ex}
\chapter{Introducere}

Acest text își propune să:

\begin{itemize}
\item enunțe cadrul de desfășurare al coomologiei étale
\item prezinte versiuni étale ale unor rezultate fundamentale în topologie, precum dualitatea Poincaré sau formulele de tip Lefschetz
\item aplice aceste rezultate la studiul funcțiilor zeta asociate varietăților peste corpuri finite
\end{itemize}

\setcounter{secnumdepth}{2}

\chapter{Situl étale}
\pagenumbering{arabic}
\setcounter{page}{1}

\begin{defi}
Un morfism de inele $A \rightarrow B$ se numește {\bf plat} dacă functorul $B \otimes_A \cdot :  A\textnormal{-Mod} \rightarrow B\textnormal{-Mod}$ este exact.
\end{defi}

\begin{defi}
Un morfism de varietăți (sau scheme) $\phi: Y \rightarrow X$ este {\bf plat} dacă morfismele locale $\mathcal{O}_{X,\phi(y)} \rightarrow \mathcal{O}_{Y,y}$ sunt plate pentru orice $y$ din $Y$.
\end{defi}

\begin{defi}
Un morfism local de inele locale $f: A \rightarrow B$ se numește {\bf neramificat} dacă $A/\mathfrak{m}_A \hookrightarrow B/f(\mathfrak{m}_A)B$ este o extindere finită și separabilă.
\end{defi}

%definiția morfismului de tip finit între varietăți

\begin{defi}
Un morfism de varietăți (sau scheme) $\phi: Y \rightarrow X$ este {\bf neramificat} dacă este de tip finit și morfismele locale $\mathcal{O}_{X,\phi(y)} \rightarrow \mathcal{O}_{Y,y}$ sunt neramificate pentru orice $y$ din $Y$.
\end{defi}

\begin{defi}
Un morfism (regulat) între două varietăți este {\bf étale} dacă este plat și neramificat.
\end{defi}

Morfismele étale au următoarele proprietăți:

\begin{prop}
\begin{enumerate}
\item Orice imersie deschisă este étală.
\item Compunerea a două morfisme étale este étală.
\item Un morfism care este schimbare de bază a unui morfism étale este étale.
\item Dacă $\phi \circ \psi$ și $\phi$ sunt étale, atunci și $\psi$ este étale.
\end{enumerate}
\end{prop}

De acum încolo vom lucra cu o varietate $X$ peste un corp algebric închis $k$.

O vecinătate étală a unui punct $x$ din $X$ este o aplicație étală $\phi: U \rightarrow X$ împreună cu un punct $u \in U$ cu $\phi(u)=x$. Un morfism de vecinătăți étale $(V,v) \rightarrow (U,u)$ este o aplicație regulată de la $V$ la $U$ care duce pe $v$ în $u$ (dacă există, este unică, din anumite proprietăți ale morfismelor étale). Am obținut astfel o categorie index și putem defini {\bf inelul local în $x$ pentru topologia étală} ca fiind:
$$\mathcal{O}_{X, \bar{x}} = \varinjlim_{(U,u)} \Gamma(U,\mathcal{O}_U)$$

Dat fiind că orice vecinătate Zariski, fiind imersie deschisă, este étală, din proprietatea limitei inductive avem un morfism natural
$$\mathcal{O}_{X, x} \rightarrow \mathcal{O}_{X, \bar{x}}$$

\chapter{De la étale la $\ell$-adic}

\chapter{Numărarea punctelor}

Problema pe care urmează să o formulăm a pornit de la cea a aflării numărului de puncte de pe o curbă eliptică, în momentul când extindem corpul de bază. Ne este cunoscută din studiul acelor curbe inegalitatea Hasse-Weil, care spune că pentru orice curbă eliptică $X$ definită peste un corp finit $\mathbb{F}_q$, dacă notăm $N_m(X)=\#X(\mathbb{F}_{q^m})$, are loc relația:
$$|N_m(X) - (q^m+1)| \leq 2\sqrt{q^m}$$

Mai precis, există două numere algebrice $\alpha_1,\alpha_2$ de modul $\sqrt{q}$ astfel încât pentru orice $m$:
$$N_m(X) = 1 - \alpha_1^m - \alpha_2^m + q^m$$

André Weil a propus următoarea generalizare:
\begin{conj}
(Weil) Fie $X$ o varietate proiectivă netedă definită peste $\mathbb{F}_q$ de dimensiune $d$. Atunci:
\begin{enumerate}
\item există $2d$ numere naturale $b_0,...,b_{2d}$ și o familie de numere complexe $\{a_{j,u}\}_{j\in\overline{0,2d}\atop s\in\overline{1,b_j}}$ astfel încât pentru orice $m$ am:
$$N_m(X)=\sum\limits_{j=0}^{2d} (-1)^j (\sum\limits_{u=1}^{b_j} \alpha_{j,u}^m)$$
Mai mult, $b_0=b_{2d}=1, \alpha_{0,1}=1, \alpha_{2d,1}=q^d$. Numărul $\sum\limits_j (-1)^j b_j$ va fi notat cu $\chi$.
\item pentru orice $j$, $b_j$ este egal cu $b_{2d-j}$, iar $(\frac{q^d}{\alpha_{2d-j,1}},...,\frac{q^d}{\alpha_{2d-j,b_j}})$ e o permutare a enumerării $(\alpha_{j,1},...,\alpha_{j,b_j})$.
\item pentru orice $j,u$, avem că $\alpha_{j,u}$ e număr algebric de modul $q^{\frac{j}{2}}$.
\end{enumerate}
\end{conj}

Se observă că dacă X este curbă eliptică se reconstituie relația de mai devreme, cu $b_1=2=dim\ H^1(\mathbb{C}/\Lambda, \mathbb{Q})$ (pentru $\Lambda$ o latice în planul complex).

Un mod mai pragmatic de a exprima conjecturile Weil este reprezentat de instrumentul funcțiilor generatoare.

Ne bazăm pe identitatea formală:
$$log(\frac{1}{1-x})=\sum\limits_{m=1}^{\infty} \frac{x^m}{m}$$
scrisă eventual
$$\frac{1}{1-x}=exp(\sum\limits_{m=1}^{\infty} \frac{x^m}{m})$$
ce se poate verifica via expansiune în serie Taylor în jurul lui zero.

Definind $Z_X(t) = exp(\sum\limits_{m=1}^{\infty} N_m(X) \frac{t^m}{m})$, obținem din punctul 1 al conjecturilor:
\begin{align*}
Z_X(t)&=exp(\sum\limits_{m=1}^{\infty} \sum\limits_{j=0}^{2d} (-1)^j \sum\limits_{u=1}^{b_j} \alpha_{j,u}^m \frac{t^m}{m})\\
&=\prod\limits_{j=0}^{2d} (\prod_{u=1}^{b_j} exp(\sum\limits_{m=1}^{\infty} \frac{(a_{j,u} t)^m}{m}))^{(-1)^j}\\
&=\prod\limits_{j=0}^{2d} (\frac{1}{\prod\limits_{u=1}^{b_j}(1-\alpha_{j,u}t)})^{(-1)^j}\\
&=\prod\limits_{j=0}^{2d} P_j(t)^{(-1)^{j+1}}
\end{align*}
unde am notat $P_j(t)=\prod\limits_{u=1}^{b_j}(1-\alpha_{j,u}t)$ (și am $P_0(t)=1-t$, $P_{2d}(t)=1-q^dt$).

Vom deriva acum din punctul 2 o relație pe care o va satisface $Z_X(t)$. Aplicăm relația de permutare între enumerări și obținem:
\begin{align*}
P_{2d-j}(t)&=\prod_u(1-\alpha_{2d-j,u}t)=\prod_u(1-\frac{q^d}{a_{j,u}}t)\\
&=(\prod_u \alpha_{j,u})^{-1} \prod_u(\alpha_{j,u}-q^dt)\\
&=(\prod_u \alpha_{j,u})^{-1} (-1)^{b_j} (q^dt)^{b_j} \prod_u (1-\frac{\alpha_{j,u}}{q^dt})\\
&=(\prod_u \alpha_{j,u})^{-1} (-1)^{b_j} (q^dt)^{b_j}P_j(\frac{1}{q^dt})
\end{align*}

Folosim acum atât simetria $b_j$-urilor cât și permutarea enumerărilor:
\begin{align*}
P_j(t)P_{2d-j}(t)&=(q^dt)^{2b_j} (q^d)^{-b_j} P_j(\frac{1}{q^dt}) P_{2d-j}(\frac{1}{q^dt})\\
&=(q^d)^{\frac{b_j+b_{2d-j}}{2}} t^{b_j+b_{2d-j}}P_j(\frac{1}{q^dt}) P_{2d-j}(\frac{1}{q^dt})\\
\end{align*}

Însă $(\prod\limits_s \alpha_{d,u})^2=(q^d)^{b_d}$, deci $\prod\limits_s \alpha_{d,u}=\pm (q^d)^{\frac{b_d}{2}}$, și pot scoate relația (pentru indicele $d$):
$$P_d(t)=\pm (-1)^{b_d} (q^dt)^{b_d} (q^d)^{\frac{b_d}{2}} P_d(\frac{1}{q^dt})$$

Și obțin astfel formula pentru funcția $Z_X$:
\begin{align*}
Z_X(t)&=\prod_{j=0}^{2d} P_j(t)^{(-1)^{j+1}}\\
&=\pm \prod_{j=0}^{2d} P_j(\frac{1}{q^dt})^{(-1)^{j+1}} (q^d)^{-\frac{\sum\limits_j (-1)^{j+1} b_j}{2}} t^{-\sum\limits_j (-1)^{j+1} b_j}\\
&=\pm q^{\frac{d\chi}{2}} t^{\chi} Z_X(\frac{1}{q^dt})
\end{align*}
numită {\bf ecuația funcțională} a lui $Z_X$.

O altă reformulare ne este dată de următoarea substituție:
$$\zeta_X(s)=Z_X(q^{-s})$$

Notând pentru un punct închis $x \in X$ cu $\kappa(x)$ corpul rezidual al său, cu $deg(x)$ gradul extinderii $\kappa(x) : \mathbb{F}_q$ și cu
\begin{align*}
N_m(x) = \left\{
     \begin{array}{lr}
       deg(x) & \text{dacă } deg(x)\ |\ m\\
       0 & \text{altfel}
     \end{array}
   \right.
\end{align*}
avem din cele cunoscute de la teoria schemelor:
$$N_m(X)=\sum\limits_x N_m(x)$$

Obținem rescrierile:
\begin{align*}
Z_X(t)&=exp(\sum_{m\geq 1} N_m(X) \frac{t^m}{m}) = exp(\sum_{m\geq 1} \sum\limits_x N_m(x) \frac{t^m}{m})\\
&=exp(\sum_x \sum_{m \geq 1 \atop deg(x) | m} N_m(x) \frac{t^m}{m})\\
&=exp(\sum_x \sum_{n \geq 1} deg(x) \frac{t^{n\cdot deg(x)}}{n \cdot deg(x)})\\
&=exp(\sum_x \sum_{n\geq 1} \frac{(t^{deg(x)})^n}{n})=exp(\sum_x log \frac{1}{1-t^{deg(x)}})\\
&=exp\ log \prod_x \frac{1}{1-t^{deg(x)}} = \prod_x \frac{1}{1-t^{deg(x)}}
\end{align*}
și deci
$$\zeta_X(s)=\prod_x \frac{1}{1-(q^{deg(x)})^{-s}} = \prod_x \frac{1}{1-(\#\kappa(x))^{-s}}$$
Ultima formulă are sens pentru o schemă oarecare, nu neapărat peste un corp finit. De pildă, înlocuind $X$ cu $\text{Spec }\mathbb{Z}$, apare:
$$\zeta_{\text{Spec }\mathbb{Z}}(s)=\prod_{\mathfrak{p} \in \text{Max }\mathbb{Z}} \frac{1}{1-(\#(\frac{\mathbb{Z}}{\mathfrak{p}}))^{-s}} = \prod_{p\text{ prim}} \frac{1}{1-p^{-s}}$$
binecunoscuta funcție zeta a lui Riemann (punctele închise din $\text{Spec }\mathbb{Z}$ sunt precis idealele maximale ale lui $\mathbb{Z}$), ceea ce ne justifică notația.

Atenție, însă: funcția $Z_X$ nu are sens decât pentru scheme definite pentru un corp finit!

Este clar că $s\in\mathbb{C}$ este zerou, respectiv pol, pentru $\zeta_X$ dacă și numai dacă $q^{-s}$ va avea aceeași calitate pentru $Z_X$ (și orice $t \in \mathbb{C}^*$ se poate scrie ca $q^{-s}$, e drept, într-o infinitate de moduri). Iau un astfel de $s$. Din exprimarea rațională a lui $Z_X$ rezultă că $q^{-s}=\frac{1}{\alpha_{j,u}}$ pentru anumite $j$ și $u$, iar ultima afirmație din conjecturi implică $|q^{-s}|=q^{-\frac{j}{2}}$. Scriu $s=a+bi$, cu $a,b\in\mathbb{R}$, și am $q^{-s}=e^{-s\ ln q} = e^{-a\ ln q- bi\ ln q}$. Deci $q^{-\frac{j}{2}}=|e^{-a\ ln q- bi\ ln q}|=q^{-a}$ și $\frac{j}{2}=a=\mathbb{R}e\ s$.

Invers, acum, dacă fac presupunerea că orice $s$ zerou sau pol al lui $\zeta_X$ are proprietatea că $\mathbb{R}e\ s=\frac{j}{2}$ (cu $j\in\overline{0,2d}$), pot face următorul raționament. Iau $t \in \mathbb{C}^*$ zerou sau pol pentru $Z_X$ și fie $s$ cu $t=q^{-s}$. Atunci $s$ este zerou/pol pentru $\zeta_X$ și $\mathbb{R}e\ s=\frac{j}{2}$. Ca înainte, avem $|t|=|q^{-s}|=q^{-\mathbb{R}e\ s}=q^{-\frac{j}{2}}$. Deci zerourile și polii lui $Z_X$ au modulul $q^{-\frac{j}{2}}$.

Rezultă că acea condiție 3 este echivalentă cu faptul că funcția zeta a varietății are zerourile și polii pe liniile $\mathbb{R}e\ s=\frac{j}{2}$, ceea ce justifică numele dat condiției de {\bf ipoteză Riemann}.

Mai mult decât atât, această separare după modul a zerourilor și polilor lui $Z_X$ ne permitm să extragem $P_j$-urile din $Z_X$ și să obținem în mod reciproc din ecuația funcțională condiția 2 din conjecturile Weil.

Toate aceste manipulări de formule ne învață să apreciem modul cum coomologia étală conduce în mod natural măcar la expresia rațională și la ecuația funcțională a funcției $Z_X$.

Să vedem cum. Mai întâi, vom obține o altă descriere a lui $X(\mathbb{F}_{q^m})$ cu ajutorul aplicației Frobenius.

Orice $\mathbb{F}_q$-algebră $A$ admite endomorfismul Frobenius $a \mapsto a^q$, ce se dualizează la o aplicație $\text{Spec } A \rightarrow \text{Spec } A$. Această familie de aplicații se prelungește unic la o întreagă transformare naturală $\{F_X : X \rightarrow X\}_{X \in \mathbb{F}_q-Sch}$. Din naturalitate, rezultă că acționează pe varietăți afine sau proiective în modul firesc, prin ridicarea la puterea $q$ a coordonatelor. În particular $F_X$ are gradul $q^{dim X}$, după cum se verifică ușor pe spațiile afine.

Rezultă deci  $X(\mathbb{F}_{q^m})=Fix(F_X^m)$.

\begin{lema}
$\Gamma_{F^m} \cdot \Delta_X = \sum\limits_{P \in Fix(F_X^m)} P$ (în sensul că toate apar cu multiplicitate $1$).
\end{lema}

\begin{proof}
E suficient să arătăm pentru $m=1$ ($F^m$ este Frobeniusul lui $\mathbb{F}_{q^m}$). Notez $F=F_X$.

Fie $P\in Fix(F)$. Înlocuiesc $X$ cu o vecinătate afină a lui $P$, să zicem $U=\text{Spec }A$, cu $A=\mathbb{F}_q[t_1,...,t_n]=\frac{\mathbb{F}_q[T_1,...,T_n]}{\mathfrak{a}}$.

Atunci pentru orice $i$ am $t_i \circ F = t_i^q$ și $(dt_i)_P \circ (dF)_P = (dt_i^q)_P = qt_i^{q-1}(dt_i)_P = 0$.

Ca urmare diferențiala lui $F$ în $P$ este zero, ca urmare graficul lui Frobenius nu este tangent la diagonală în $(P,P)$, iar numărul de intersecție (multiplicitatea) este $1$. 
\end{proof}

Aplicând formula de punct fix a lui Lefschetz, rezultă (pentru $(\ell,q)=1$):
$$N_m(X) = \sum\limits_r (-1)^r Tr(F^m_{| H^r(X,\mathbb{Q}_\ell)})$$

Putem folosi formula pentru a prelucra funcția $Z$ a lui $X$:
\begin{align*}
Z_X(t)&=exp(\sum\limits_{m=1}^{\infty} N_m(X) \frac{t^m}{m})\\
&=exp(\sum\limits_{m=1}^{\infty} (\sum\limits_{r=0}^{2d} (-1)^r Tr(F^m_{| H^r(X,\mathbb{Q}_\ell)}) \frac{t^m}{m})\\
&=\prod\limits_{r=0}^{2d}exp(\sum\limits_{m=1}^{\infty}Tr(F^m_{| H^r(X,\mathbb{Q}_\ell)})  \frac{t^m}{m}) ^ { (-1)^r }
\end{align*}

Scriem acum fiecare $F_{| H^r(X,\mathbb{Q}_\ell)}$ ca matrice pătratică superior triunghiulară (eventual peste o extindere a lui $\mathbb{F}_q$) cu numărul de linii egal cu $b_r=dim\ H^r(X,\mathbb{Q}_\ell)$. Se observă atunci că dacă elementele de pe diagonală (valorile proprii) sunt $\alpha_{r,1},...,\alpha_{r,b_r}$, atunci $Tr(F^m_{| H^r(X,\mathbb{Q}_\ell)})$ va fi egal cu $\sum\limits_{i=1}^{b_r} \alpha_{r,i}^m$ și deci pot aplica exact același raționament ca mai devreme pentru a exprima rațional pe $Z_X$.

Vom avea $Z_X \in \mathbb{Q}_\ell(t) \cap \mathbb{Q}[[t]] \subseteq \mathbb{Q}(t)$, însă aceasta nu ne garantează că fiecare $P_j$ este în $\mathbb{Q}[t]$ (altfel spus, că este „independent de $\ell$”) - aceasta se poate face doar presupunând ipoteza lui Riemann, care ne permite, după cum am spus și mai devreme, să separăm $P_j$-urile după modulul rădăcinilor.

Altă consecință a raționamentului precedent a fost că am identificat $\alpha_{j,u}$-urile ca fiind valorile proprii ale operatorilor induși de Frobenius pe spațiile de coomologie.

Aceasta ne arată în particular că $b_0=b_{2d}=1$.

Trecem acum la demonstrarea simetriei între valorile proprii pe spațiile de ordin $r$ și $2d-r$, relație care după cum am observat ne implică o ecuație funcțională.

Avem biliniara nedegenerată dată de dualitatea Poincaré:
$$\langle,\rangle=\eta_X\ \circ \smile\ : H^{2d-r}(X,\mathbb{Q}_\ell) \times H^{r}(X,\mathbb{Q}_\ell) \rightarrow H^{2d}(X,\mathbb{Q}_\ell) \simeq \mathbb{Q}_\ell(-d)$$
ce din start ne indică $b_r=b_{2d-r}$.

Știm că $F^*_r=F_{| H^r(X,\mathbb{Q}_\ell)}$ are adjunct relativ la $\langle,\rangle$:
$$\langle F_{*2d-r}(x),x'\rangle=\langle x,F^*_r(x')\rangle, \forall\ x \in H^{2d-r}(X,\mathbb{Q}_\ell), x' \in H^{r}(X,\mathbb{Q}_\ell)$$

Din considerente de algebră liniară rezultă că valorile proprii ale lui $F^*_r$ coincid cu ale lui $F_{*2d-r}$. Însă $F_{*r} \circ F^*_r = q^d (=deg\ F)$.

De aici rezultă că dacă $(\alpha_1,...,\alpha_v)$ sunt valorile proprii ale lui $F^*_r$, $(\frac{q^d}{\alpha_1},...,\frac{q^d}{\alpha_v})$ sunt cele ale lui $F_{*r}$, deci (din cele anterioare) și ale lui $F^*_{2d-r}$, i.e. exact ce ni se cerea. Iar faptul că $\alpha_{0,1}=1$ și $\alpha_{2d,1}=q^d$ rezultă din modul cum acționează operatorii Frobenius pe $H^0$, respectiv pe $H^{2d}$.

În acest moment, tot ce ne rămâne este să demonstrăm ipoteza lui Riemann.

\chapter{«La conjecture de Weil»}

Mai precis, ce avem de demonstrat este:

\begin{teo}
(Deligne, 1974) Fie $X$ o varietate proiectivă $d$-dimensională absolut nesingulară și absolut ireductibilă definită peste $\mathbb{F}_q$; $\alpha$ o valoare proprie a lui $F^*_r$; $\tau$ o scufundare a lui $\mathbb{Q}_\ell$ în $\mathbb{C}$.

Atunci $\tau(\alpha)$ (mai departe simbolul $\tau$ va fi subînțeles) este algebric și de modul $q^{\frac{r}{2}}$.
\end{teo}

În primul rând, se observă că este suficient să demonstrăm pentru varietatea obținută după o schimbare de bază spre o extindere finită a corpului de definiție - să zicem, de grad $m$. Asta deoarece operatorul Frobenius pe varietatea nouă va fi puterea $m$ a celui de pe varietatea veche. Dacă $\alpha$ este valoare proprie pentru Frobenius-ul vechi, $\alpha^m$ este pentru cel nou. Din ipoteza noastră, $\alpha^m$ are modulul $(q^{m})^{\frac{r}{2}}$. Ca urmare, $\alpha$ va avea modulul $q^{\frac{r}{2}}$. Aceasta ne va permite să facem un număr finit de extinderi finite de-a lungul demonstrației, fără a pierde din generalitatea enunțului.

În al doilea rând, se vede că pot demonstra doar pentru spațiile cu rangul cel mult $d$. Aceasta deoarece de la rangul $d+1$ încolo, valorile proprii au simetria implicată de dualitatea Poincaré pe care am văzut-o mai devreme.

În al treilea rând, vom arăta că este suficent să arătăm teorema doar pentru $r=d$. Iată de ce: din teorema lui Bertini, există $Z \subset X$ o secțiune hiperplană netedă (eventual extinzând corpul).  Aplicând teorema Lefschetz slabă, aplicația canonică (ce este compatibilă cu Frobenius) $H^r(X,\mathbb{Q}_\ell) \rightarrow H^r(Z,\mathbb{Q}_\ell)$ este injectivă pentru $r \leq d - 1$ și pot aplica un raționament prin inducție (pasul de bază fiind varietățile zero-dimensionale, pentru care clar este adevărat, din modul cum acționează Frobenius pe $H^0$).

În al patrulea rând, putem să arătăm chiar și numai pentru varietățile de dimensiune pară, iar pentru acelea, doar că valorile proprii $\alpha$ corespunzătoare spațiului de coomologie din mijloc satisfac inegalitatea
$$q^{\frac{d}{2}-\frac{1}{2}}\leq|\alpha|\leq q^{\frac{d}{2}+\frac{1}{2}}$$
Presupunând că am arătat așa ceva, vreau să demonstrez teorema pentru o varietate oarecare $X$ și $\alpha$ valoare proprie a lui $F^*_d$. Fie $k$ număr natural. Iau $Y$ ca fiind produsul lui $X$ cu el însuși de $2k$ ori. Conform Künneth, $H^d(X,\mathbb{Q}_\ell)^{\otimes 2k}$ se scufundă în $H^{2kd}(Y,\mathbb{Q}_\ell)$, iar $\alpha^{2k}$ va fi valoare proprie a lui Frobenius aplicat pe spațiul de pe urmă. Ca urmare, va avea loc:
$$q^{\frac{2kd}{2}-\frac{1}{2}}\leq|\alpha|^{2k}\leq q^{\frac{2kd}{2}+\frac{1}{2}}$$
Scoțând radical de ordin $2k$, obțin:
$$q^{\frac{d}{2}-\frac{1}{4k}}\leq|\alpha|\leq q^{\frac{d}{2}+\frac{1}{4k}}$$
Cum relația are loc pentru $k$ arbitrar, trecându-l la infinit, obțin $|\alpha|=q^{\frac{d}{2}}$.

De acum înainte, prin urmare, $d$ va fi par.

În acest moment, după ce am încheiat reducerile geometrice de mai devreme, putem trece la miezul problemei și să introducem tehnica numită {\it pencil Lefschetz}.

Luăm o scufundare a lui $X$ într-un $\mathbb{P}^N$ și luăm $L$ un subspațiu liniar proiectiv de codimensiune $2$ ce intersectează transversal pe $X$. Mulțimea hiperplanelor din $\mathbb{P}^N$ ce conțin pe $L$ are ca spațiu de moduli pe $\mathbb{P}^1$ și deci poate fi organizată ca o familie $\{H_d\}_{d\in\mathbb{P}^1}$. Iau apoi mulțimea:
$$\widetilde{X}=\{(x,d)\in X \times \mathbb{P}^1 \mid x \in H_d\}$$
ce are structură de varietate algebrică, anume {\it eclatarea lui $X$ în $L\cap X$} (este netedă, din faptul că $L$ intersectează transversal pe $X$). Ea este înzestrată cu două aplicații canonice:
$$X \leftarrow \widetilde{X} \xrightarrow{f} \mathbb{P}^1$$
iar din teoria eclatării, aplicația $H^d(X,\mathbb{Q}_\ell) \rightarrow H^d(\widetilde{X},\mathbb{Q}_\ell)$ este injectivă. Putem trece deci de la $X$ la $\widetilde{X}$ fără probleme.



















\bibliographystyle{plain}
\bibliography{etale}
\end{document}
