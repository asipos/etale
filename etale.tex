\documentclass[13pt,openany]{book}
\usepackage[romanian]{babel}
\usepackage{euler}
\usepackage{xltxtra}
\usepackage{color}
\usepackage{indentfirst}
\usepackage{mathtools}
\usepackage{amsmath}
\usepackage{amsfonts}
\usepackage{amsthm}
\usepackage{amssymb}
\usepackage{listings}
\usepackage{enumerate}
\usepackage[margin=1.1in]{geometry}
\usepackage[linktoc=none]{hyperref}
\usepackage{url}
\usepackage[nottoc]{tocbibind}
\usepackage[raggedright]{titlesec}
\setmainfont{URW Palladio L}
\setcounter{secnumdepth}{-1}
\setcounter{chapter}{0}

\begin{document}

\newtheorem{teo}{\bf Teorema}[chapter]
\newtheorem{cor}[teo]{\bf Corolarul}
\newtheorem{lema}[teo]{\bf Lema}
\newtheorem{prob}[teo]{\bf Problema}
\newtheorem{conj}[teo]{\bf Conjectura}
\newtheorem{propr}[teo]{\bf Proprietatea}
\newtheorem{prop}[teo]{\bf Propoziția}
\newtheorem{alg}[teo]{\bf Algoritmul}
\theoremstyle{remark}
\newtheorem{obs}[teo]{\bf Observația}
\newtheorem{obss}[teo]{\bf Observațiile}
\newtheorem{ex}[teo]{\bf Exemplul}
\newtheorem{exs}[teo]{\bf Exemplele}
\theoremstyle{definition}
\newtheorem{defi}[teo]{\bf Definiția}

\newcommand{\calm}{\mathcal{M}}
\newcommand{\Qell}{\mathbb{Q}_\ell}

\pagenumbering{roman}
\setcounter{page}{1}
\fontsize{4.3mm}{6mm}\selectfont
\pagestyle{empty}
\begin{center}

\LARGE UNIVERSITATEA DIN BUCUREȘTI\\[0.5cm]
\LARGE Facultatea de Matematică și Informatică\\[3cm]

\Large Disertație masterală\\[3.5cm] 
 
\rule{8cm}{0.5mm}\\[0.8cm]
{ \huge \bfseries Geometria estimărilor}\\[0.2cm]
\rule{8cm}{0.5mm}\\[2.5cm]

\begin{minipage}{0.4\textwidth}
\begin{flushleft} \large
\emph{Autor:}\\
Andrei Sipoș
\end{flushleft}
\end{minipage}
\begin{minipage}{0.5\textwidth}
\begin{flushright} \large
\emph{Profesor coordonator:} \\
Conf. dr. Victor Vuletescu
\end{flushright}
\end{minipage}

\vfill

{\large București, 2014}

\end{center}
\newpage
\phantom{X}
\newpage

\tableofcontents

\newpage
\phantom{X}
\newpage

\setlength{\parskip}{1.5ex plus 0.5ex minus 0.2ex}
\chapter{Introducere}

Acest text își propune să:

\begin{itemize}
\item enunțe cadrul de desfășurare al coomologiei etale
\item prezinte versiuni etale ale unor rezultate fundamentale în topologie, precum dualitatea Poincaré sau formulele de tip Lefschetz
\item aplice aceste rezultate la studiul funcțiilor zeta asociate varietăților peste corpuri finite
\end{itemize}

\cite{milne}\cite{lovering}\cite{dugger}

\newpage
\phantom{X}
\newpage

\pagestyle{headings}
\setcounter{secnumdepth}{2}

\chapter{Situl etal}
\pagenumbering{arabic}
\setcounter{page}{1}

Orice schemă este înzestrată cu un morfism canonic către Spec $\mathbb{Z}$, acesta fiind obiectul terminal al categoriei. Să presupunem însă că imaginea unei scheme prin acel morfism unic este doar un punct, să zicem $(p)$, cu $p$ număr prim. În acest caz, fasciculul structural al schemei va avea o structură naturală de fascicul de $\mathbb{F}_p$-algebre, iar morfismul va putea fi „corestricționat” la un morfism către Spec $\mathbb{F}_p$.

Pe baza acestor idei, putem construi următoarea categorie: obiectele sunt perechi $(T,\phi)$ cu $T$ schemă și $\phi: T \rightarrow$ Spec $\mathbb{F}_p$ morfism de scheme, iar morfismele între două obiecte $(T,\phi)$ și $(T',\phi')$ sunt morfismele $\psi: T \rightarrow T'$ cu proprietatea că $\phi'\circ\psi=\phi$. Această categorie „virgulă” ({\it comma category}) o vom numi {\bf categoria schemelor peste $\mathbb{F}_p$}. Iar construcția, firește, o putem face luând orice schemă $S$ în loc de Spec $\mathbb{F}_p$, categoria rezultantă numindu-se {\bf categoria $S$-schemelor}.

Acest „punct de vedere local” sau „relativ”, cum l-a numit A. Grothendieck, va fi folosit exhaustiv în cele ce urmează, în sensul că, deși ne vom concentra atenția asupra unei scheme $S$, o vom face într-un mod extrinsec, considerând morfismele, ca mai sus, ce au pe $S$ drept codomeniu, însă numai pe acelea ce satisfac o oarecare noțiune de „acoperire”. Noțiunea precisă va fi aceea de {\bf morfism etal}.

Un morfism etal între două scheme este un morfism plat și neramificat. Platitudinea este definită la nivel local: un {\bf morfism plat} de scheme în sensul că morfismele induse între inelele locale sunt plate în sensul algebric (induc functori exacți la nivelul categoriilor de module). Ca un morfism să fie {\bf neramificat}, el trebuie să fie {\bf de tip finit} (adică luând preimaginea oricărui deschis afin, ea admite o acoperire finită cu deschiși afini astfel încât morfismele de inele induse de deschișii acoperirii sunt toate de tip finit) și să satisfacă următoarea proprietate definită tot la nivel local: morfismele induse între inelele locale $\mathcal{O}_x \rightarrow \mathcal{O}_{f(x)}$ fac ca extinderile induse de corpuri $\frac{\mathcal{O}_x}{\mathfrak{m}_{\mathcal{O}_x}} \hookrightarrow \frac{\mathcal{O}_{f(x)}}{f(\mathfrak{m}_{\mathcal{O}_x})\mathcal{O}_{f(x)}}$ să fie finite și separabile. (Pentru un morfism neramificat $X \rightarrow Y$, aplicația canonică $X \rightarrow X\times_Y X$ are imaginea deschisă.)

Morfismele etale se comportă bine cu compunerea, în două moduri. Primul este cel evident - compunerea a două morfisme etale este etală. Al doilea este oarecum mai subtil - dacă fixăm o schemă $S$ și ne uităm la subcategoria $S$-schemelor etale (în sensul că $\phi$-ul de mai devreme este morfism etal), atunci toate morfismele din această subcategorie sunt și ele etale - legătura cu compunerea este dată de faptul că proprietatea se poate exprima și: {\it dacă $f$ și $g$ sunt morfisme de scheme astfel încât $g$ și $g \circ f$ sunt etale, atunci și $f$ este etal}.

Am ajuns acum la enunțarea primului rezultat semnificativ al teoriei, ce corespunde oarecum teoremei clasice din topologie a unicității ridicării drumului într-o acoperire. Mai întâi, însă, să definim noțiunea de punct geometric. Fie $x$ un punct al unei scheme $X$. Fixăm o închidere algebrică $K$ a corpului rezidual în $x$ ce induce un morfism $\bar{x}=$ Spec $K\rightarrow X$. Numim un astfel de morfism {\bf punct geometric}.

\begin{lema}
(Lema de rigiditate) Fie $X$, $Y$ două $S$-scheme etale finite, astfel încât $X$ este conexă și $Y$ separată, și $f$ și $g$ două morfisme de la $X$ la $Y$. Fie $\bar{x} \rightarrow X$ un punct geometric, astfel încât $f(\bar{x})=g(\bar{x})$. Atunci $f=g$.
\end{lema}

\begin{proof}
Dat fiind că aplicația $Y \rightarrow S$ este etală și separabilă, $\Delta : Y \rightarrow Y \times_S Y$ este deschisă și închisă. Ca urmare putem scrie codomeniul ca reuniune disjunctă de scheme între imagine și o schemă $Z$. Însă imaginea aplicației $f \times_S g: X \rightarrow Y \times_S Y$ taie imaginea lui $\Delta$ în $(f(\bar{x}),g(\bar{x}))$, și ca urmare (cum $X$ este conexă) este conținută pe de-a-ntregul în imaginea lui $\Delta$. Deci $f=g$.
\end{proof}

Această lemă ne conduce imediat către următorul raționament: dacă $X$ este o $S$-schemă finită etală conexă, $\bar{x} \rightarrow S$ un punct geometric și $\bar{x}'\rightarrow X$ o secțiune a sa - mulțimea acestor secțiuni o notăm cu $X_{\bar{x}}$ - aplicația care duce un $S$-automorfism $f$ al lui $X$ în $f(\bar{x})$ este clar injectivă. Spunem că $X$ este o $S$-schemă {\bf Galois} (sau că morfismul $X \rightarrow S$ este Galois) dacă această aplicație este chiar bijectivă.

De acum încolo, considerăm $S$ conexă. Prin analogie cu teoria clasică Galois, avem următorul rezultat.

\begin{prop}
Fie $X \rightarrow S$ aplicație Galois finită și $G=Aut_S(X)$ grupul $S$-automorfismelor lui $X$. Aplicațiile $Y \mapsto Aut_Y(X)$ și $H \mapsto X^H$ realizează o echivalență de categorii între subacoperirile etale ale lui $X \rightarrow S$ și subgrupurile lui $G$ (această ultimă categorie este la rândul ei echivalentă cu cea a $G$-mulțimilor tranzitive). În plus, dacă $X \rightarrow S$ este o aplicație finită doar etală, există și este unic (până la izomorfism) $X' \rightarrow X$ ce acoperă Galois pe $S$ în mod universal relativ la $X$ („închiderea Galois” a lui $X \rightarrow S$).
\end{prop}

Să privim această corespondență dintr-un alt punct de vedere. Dacă $S$ este o schemă, notăm cu $FEt(S)$ categoria $S$-schemelor finite etale. Fie $\bar{x} \rightarrow S$ un punct geometric. Atunci:
$$Fib_{\bar{x}} : FEt(S) \rightarrow FinSet$$
ce acționează ca $(X \rightarrow S) \mapsto \text{Hom}_S(\bar{x},X)=X_{\bar{x}}$ este un functor, numit {\bf functorul fibră al lui Grothendieck}. Vom nota cu $\pi_1(S,\bar{x})$ și vom numi grupul fundamental etal grupul de automorfisme ale acestui functor.

O identificare mai concretă a grupului se poate face într-un mod asemănător cu spațiile de acoperire din topologia algebrică clasică. Fixăm o schemă $S$ și un punct geometric al ei $\bar{x}$. Consider familia $\{X_\alpha\}_\alpha$ a tuturor acoperirilor finite Galois ale lui $S$, ce are structură de diagramă în categoria $S$-schemelor. Pun în fiecare $X_\alpha$ câte un punct geometric $\bar{x}_\alpha$ astfel încât alegerile făcute să fie compatibile cu morfismele diagramei. Atunci putem arăta:
$$Fib_{\bar{x}} \simeq \varinjlim \text{Hom}_S(X_\alpha,\cdot)$$
Pentru a construi morfismul de la dreapta la stânga, trebuie să dăm câte un morfism $\phi_{\alpha,Y} : \text{Hom}_S(X_\alpha,Y) \rightarrow Fib_{\bar{x}}(Y)$ pentru orice $\alpha,Y$. Aceasta se face punând $\phi(f)=f(\bar{x}_\alpha)$. Invers, luând o schemă $Y$ și $\bar{y}$ în $Fib_{\bar{x}}(Y)$, aleg o închidere Galois a lui $Y$ și factorizez $\bar{y} \rightarrow X_\alpha \rightarrow Y$. Atunci există un unic $S$-automorfism al lui $X_\alpha$ care duce pe $\bar{x}_\alpha$ în $\bar{y}$, ce dă un element al limitei inductive. Izomorfismul este deci arătat.

Ca o consecință, avem descrierea:
$$\pi_1(S,\bar{x}) \simeq \varprojlim Aut_S(X_\alpha)^{op}$$

\begin{ex}
Dacă avem un corp $k$, a alege un punct geometric al său $\bar{x}$ se rezumă la a fixa o închidere algebrică. Putem acum calcula:
$$\pi_1(\text{Spec }k,\bar{x})=\varprojlim\limits_{L \text{ ext. finită separabilă}} Gal(\frac{L}{k}) = Gal(\frac{k^s}{k})$$
adică grupul Galois absolut al corpului $k$.
\end{ex}

Prin urmare, grupurile $\pi_1(S,\bar{x})$ sunt profinite și acționează continuu pe $Fib_{\bar{x}}(X)$ pentru orice $X \rightarrow S$ finită etală, iar orbitele acțiunilor sunt componentele conexe ale fibrelor. Functorul $Fib_{\bar{x}}$ poate fi deci privit ca fiind definit pe $FEt(S)$ și luând valori în categoria $\pi_1(S,\bar{x})$-mulțimilor la stânga, finite și continue. Am ajuns acum să putem reexprima în acest cadru corespondența de mai devreme:

\begin{teo}
Functorul $Fib_{\bar{x}}$, redefinit ca mai înainte, reprezintă o echivalență de categorii.
\end{teo}

\begin{proof}
În primul rând, să arătăm că este esențial surjectiv. Fie $\Sigma$ o $\pi_1(S,\bar{x})$-mulțime finită și continuă. Pentru simplitate, o presupunem tranzitivă. Pentru orice $x$ din $\Sigma$, stabilizatorul său este deschis, deci conține un element al unei baze de deschiși, în particular un subgrup normal al cărui cât este de forma $G=Aut_S(P)$, cu $P \rightarrow S$ acoperire Galois. Ca urmare $\pi_1(S,\bar{x})$ acționează printr-un asemena cât finit, iar imaginea în $G$ a stabilizatorului conduce via corespondența de mai devreme la o $S$-schemă finită etală $X$, care din lema de rigiditate are fibra izomorfă cu $(G/H)_s$, care din teorema orbită-stabilizator este izomorfă cu $\Sigma$.

Demonstrăm acum că functorul este deplin fidel. Lema de rigiditate ne indică imediat că este fidel, așa că rămâne de arătat doar surjectivitatea pe Hom-set-uri. Fără a restrânge generalitatea, presupunem $X$ și $Y$ conexe. Consider un morfism de $\pi_1(S,\bar{x})$-mulțimi $f:X_{\bar{x}} \rightarrow Y_{\bar{x}}$. Atunci stabilizatorul lui $\bar{x}'$ este inclus în stabilizatorul lui $f(\bar{x}')$, pentru orice $\bar{x}'$ din $X_{\bar{x}}$. Alegem o închidere Galois $P \rightarrow X$ și observăm că orice $\sigma$ ce acționează trivial pe $P$ stabilizează și pe $\bar{x}'$, și pe $f(\bar{x}')$. Reducându-ne la câtul finit de $X$-automorfisme ale lui $P$, aplicăm iar corespondența de mai devreme pentru a obține morfismul dorit $X \rightarrow Y$.
\end{proof}

\begin{prop}
Fie $S$ o schemă conexă și $\bar{x}$, $\bar{x}'$ două puncte geometrice ale sale. Atunci există un izomorfism între functorii fibră $Fib_{\bar{x}}$ și $Fib_{\bar{x}'}$.
\end{prop}

\begin{proof}
Știm de mai devreme că fiecare functor fibră se scrie ca limită de reprezentabile, în a cărei diagramă obiectele coincid de la un functor la altul, însă diferă funcțiile de „tranziție”. Pentru a construi un izomorfism între functori este suficient să dăm un izomorfism între diagrame. Luăm $X_\alpha$ și $X_\beta$ două acoperiri Galois ale lui $S$ cu funcțiile de tranziție $\phi_{\alpha\beta}$ și $\psi_{\alpha,\beta}$, pentru $\bar{x}$ și respectiv $\bar{x}'$. Fie punctele bază peste $\bar{x}$ ale acoperirilor $p_\alpha$ și $p_\beta$. Din construcție, $Fib_{\bar{x}}(\phi_{\alpha\beta})(p_\beta)=p_\alpha$. Luăm $\lambda_\beta$ un $S$-automorfism al lui $P_\beta$ și fie $p'_\alpha:=Fib_{\bar{x}}(\psi_{\alpha\beta})(\lambda_\beta(p_\beta))$. Dat fiind că $P_\alpha$ este Galois, există un unic $S$-automorfism $\lambda_\alpha$ al lui $P_\alpha$ care duce $p_\alpha$ în $p'_\alpha$. Rezultă că $\lambda_\alpha \circ \phi_{\alpha\beta} = \psi_{\alpha\beta} \circ \lambda_\beta$. Această construcție induce izomorfismul de diagrame cerut.
\end{proof}

\begin{cor}
Dacă $S$ este o schemă conexă și $\bar{x}$, $\bar{x}'$ două puncte geometrice ale sale. Atunci există un izomorfism continuu de grupuri profinite $\pi_1(S,\bar{x}) \rightarrow \pi_1(S,\bar{x}')$.
\end{cor}

Începe deja să se vadă analogia cu grupul fundamental clasic: acolo un izomorfism între grupurile asociate aceluiași spațiu cu puncte bază diferite era indus de un drum între cele două puncte. Aici, analogul drumului este dat de izomorfismul între cei doi functori fibră, care din aceasta similaritate, este numit {\bf drum} între punctele geometrice.

Dacă avem o schemă conexă $W$ de tip finit (peste Spec $\mathbb{Z}$), toate punctele sale închise au corpuri reziduale finite, astfel că aplicațiile canonice Spec $\kappa(x) \rightarrow W$ vor induce morfisme $\pi_1(\text{Spec }\kappa(x),\bar{x}) \rightarrow \pi_1(W,\bar{x})$. Însă primul grup este izomorf cu completarea profinită a lui $\mathbb{Z}$, ce are un generator topologic canonic (morfismul Frobenius), iar imaginea sa este un element canonic în $ \pi_1(W,\bar{x})$.

Însă dacă $\bar{w}$ este un alt punct bază pentru $W$, pot din izomorfismul de mai devreme $ \pi_1(W,\bar{x}) \rightarrow \pi_1(W,\bar{w})$ să obțin un element canonic în $\pi_1(W,\bar{w})$, modulo conjugare. Clasa elementului se va numi clasa Frobenius a lui $\bar{x}$. Astfel că s-a construit pentru orice $\bar{w}$ o aplicație de la punctele închise ale lui $W$ la clasele de conjugare din $\pi_1(W,\bar{w})$.

Precum grupul fundamental clasic, și acest „$\pi_1$ etal” are un comportament functorial. Dacă avem un morfism de scheme $f: S' \rightarrow S$, atunci functorul de schimbare a bazei $BC_{S'\rightarrow S}$ duce $S$-scheme finite etale în $S'$-scheme finite etale. Iar dacă $\bar{x}'$ este un punct geometric al lui $S'$, din proprietatea de pull-back avem:
$$\text{Hom}_{S'}(\bar{x}',X \times_S S')\simeq \text{Hom}_S(\bar{x},X)$$
Ca urmare $Fib_{\bar{x}} = Fib_{\bar{x}'} \circ BC_{S'\rightarrow S}$, de unde se obține o aplicație naturală:
$$f_* : \pi_1(S',\bar{x}') \rightarrow \pi_1(S,\bar{x})$$
care induce un functor $Res_{f_*}$ de la $\pi_1(S,\bar{x})$-mulțimile finite continue la $\pi_1(S',\bar{x}')$-mulțimile finite continue. Chiar din definiție, rezultă următoarea compatibilitate:
$$Res_{f_*} \circ Fib_{\bar{x}} = Fib_{\bar{x}'} \circ BC_{S'\rightarrow S}$$

Introducem o ultimă proprietate a acestui grup fundamental etal.

\begin{teo}
Fie $X_0$ o schemă geometric conexă și quasi-compactă peste un corp perfect $k$, $X$ schema obținută prin schimbarea de bază spre o închidere algebrică $\bar{k}$, iar $\bar{x} : \text{Spec }\bar{k} \rightarrow X$ un punct geometric. Atunci există un șir exact:
$$1 \rightarrow \pi_1(X,\bar{x}) \rightarrow \pi_1(X_0,\bar{x}) \rightarrow Gal(\bar{k}/k) \rightarrow 1$$
\end{teo}

Notăm categoria morfismelor etale către o schemă fixată $S$ cu $Et(S)$.

\begin{defi}
Un {\bf prefascicul etal} (de mulțimi) pe $S$ este un functor $\mathcal{F} : Et(S)^{op} \rightarrow Set$.
\end{defi}

\begin{defi}
Un {\bf fascicul etal} (de mulțimi) este un prefascicul etal cu proprietatea că pentru orice $U \rightarrow S$ morfism etal și orice $\{U_i \rightarrow U\}_i$ familie de morfisme etale ale căror imagini acoperă împreună pe $U$ am că diagrama:
$$\mathcal{F}(U) \rightarrow \prod_i \mathcal{F}(U_i) \rightrightarrows \prod_{i,j} \mathcal{F}(U_i \times_U U_j)$$
este egalizator în $Set$.
\end{defi}

În acest moment, comparația cu noțiunile clasice din studiul varietăților, oarecum întrezărită de pe la grupul fundamental, a devenit inevitabilă, așa că se cuvine să o etalăm explicit. Dacă pentru prefascicule am transpus exact definiția clasică, înlocuind categoria indusă de mulțimea ordonată a deschișilor unui spațiu topologic cu acea categorie a morfismelor etale, la fascicule nu am mai putut să folosim exclusiv structura de categorie, dat fiind că avem nevoie să știm ce familii de morfisme reprezintă acoperiri. Categoria $Et(S)$ devine deci, împreună cu această structură în plus, ceea ce se zice un {\bf sit}, numit {\bf situl etal} al schemei $S$. (De aici și explicația pentru titlul capitolului.) Această abstractizare merge mai departe către {\it teoria toposurilor}, asupra căreia vom păstra tăcerea aici.

În practică, însă, ne este greu, folosind doar definiția, să verificăm condiția de fascicul. De aceea ne va fi de folos următorul rezultat.

\begin{prop}
\begin{enumerate}[i)]
\item Orice prefascicul reprezentabil (de forma $\widetilde{V} = \text{Hom}_S(\cdot,V) \in Set^{Et_S^{op}}$) este fascicul.
\item Condiția de mai sus este locală, în sensul că un prefasicul pe $S$ este reprezentabil dacă și numai dacă pentru orice $\bar{x} \rightarrow S$, punct geometric, există $U \rightarrow S$ vecinătate a lui astfel încât restricția prefascicului la $U$ e un prefascicul reprezentabil.
\end{enumerate}
\end{prop}

De pildă, fasciculul constant de fibră $\Sigma$ peste $S$ poate fi gândit în această accepțiune ca fiind reprezentat de $\Sigma \times S$, adică o reuniune disjunctă de $\Sigma$ cópii ale lui $S$.

Toate construcțiile cunoscute de la fasciculele pe spații topologice au un corespondent peste situl etal. La fel, subcategoria de fascicule este reflectivă și avem adjunctul de fasciculizare. Imaginea inversă a unui fascicul este construită tot luând limita inductivă și fasciculizând - de pildă, când morfismul pe care tragem fasciculul înapoi este un punct geometric, caz în care imaginea inversă o numim fibră. Analog, fasciculizarea păstrează fibrele.

Revenind la problema clasificării fasciculelor, manifestate adineauri printr-o echivalență de categorii, ne vom uita la propoziția anterioară și o vom exploata într-o oarecare măsură. Avem schema noastră de bază $S$ și $X \rightarrow S$ o acoperire etală a sa. Ea este în particular o $S$-schemă, deci ne dă un fascicul. Ne întrebăm ce fascicule se obțin în felul acesta. Găsim o clasă nouă de fascicule, denumite {\bf fascicule constructibile}, adică cele pentru care orice punct geometric $\bar{x} \rightarrow S$ admite o vecinătate etală astfel încât fasciculul restricționat la acea vecinătate este fascicul constant de fibră finită.

\begin{teo}
Acest functor reprezintă o echivalență de categorii între acoperirile finite etale ale lui $S$ și fasciculele constructibile pe $S$.
\end{teo}

(Demonstrația se folosește de tehnici de {\it teoria descentului}.)

\begin{cor}
Avem o echivalență de categorii între fasciculele constructibile pe $S$ și $\pi_1(S,\bar{x})$-mulțimile finite continue, dată chiar de functorul de luare a fibrei în $\bar{x}$.
\end{cor}

Pentru a defini, la fel cum o facem când lucrăm cu fascicule pe spații topologice, fascicule de grupuri abeliene, nu avem altceva de făcut decât să înlocuim peste tot mulțimile cu grupuri. Echivalent, putem spune că un fascicul de grupuri abeliene pe un sit etal este un obiect de tip grup abelian în categoria fasciculelor de mulțimi de pe acel sit, obținând astfel o categorie echivalentă cu cea definită anterior („comutativitatea abstracțiunii”!).

Această categorie, a fasciculelor de grupuri abeliene pe un sit etal asociat unei varietăți X, categorie notată de acum înainte cu $Ab(X)$, este abeliană și are suficiente obiecte injective. (Putem face aceeași construcție și pentru prefascicule, obținând o categorie abeliană $PreshAb(X)$.) Demonstrația dată de Grothendieck acestui ultim fapt se bazează, firește, pe existența anvelopei injective a unui grup abelian arbitrar.

În aceste condiții, cum $\Gamma: Ab(X) \rightarrow Ab$ este un functor exact la stânga, el admite functori derivați la dreapta. Vom nota $(R^r\Gamma)(\mathcal{F})$ cu $H^r(X,\mathcal{F})$ și îl vom numi al $r$-lea grup de {\bf coomologie etală} corespunzător varietății $X$.

Imediat, din teoria functorilor derivați, rezultă că $H^0$ este $\Gamma$ și că un șir exact scurt de fascicule etale abeliene induce un șir exact lung în coomologia lor.

O construcție înrudită (chiar o generalizare, după cum vom vedea) este derivarea imaginii directe a unui fascicul. Dacă $\pi: Y \rightarrow X$ este un morfism, iar $\mathcal{F}$ este un fascicul etal pe $Y$, definim $\pi_*(\mathcal{F})$ ca fiind fasciculul etal pe $X$ care verifică:
$$(\pi_*\mathcal{F})(U)=\mathcal{F}(U\times_X Y)$$
(bine definit, deoarece o schimbare de bază a unui morfism etal este etal). Această construcție este chiar un functor $\pi_*: Ab(Y) \rightarrow Ab(X)$, ce e exact la stânga, deci admite functori derivați notați cu $R^r\pi_*$.

\begin{prop}
$R^r\pi_*(\mathcal{F})$ este fasciculizatul lui $U \mapsto H^r(U \times_X Y, \mathcal{F})$.
\end{prop}

\begin{proof}
Definim $\pi_P: PreshAb(Y) \rightarrow PreshAb(X)$ ca fiind analogul lui $\pi_*$ pentru prefascicule. Atunci $\pi_* =\  \widetilde{}\ \circ \pi_P \circ i$, unde $\ \widetilde{}\ $ (fasciculizarea) și $\pi_P$ sunt exacte.

Fie $\mathcal{F} \rightarrow \mathcal{I}^\bullet$ o rezoluție injectivă. Atunci:
\begin{align*}
R^r\pi_*(\mathcal{F}) &= H^r(\pi_*\mathcal{I}^\bullet)\\
&= H^r (\ \widetilde{}\ \circ \pi_P \circ i\mathcal{I}^\bullet)\\
&= (\ \widetilde{}\ \circ \pi_P)(H^r(i\mathcal{I}^\bullet))\\
&=(\ \widetilde{}\ \circ \pi_P)(U \mapsto H^r(U, \mathcal{F}))\\
&=\ \widetilde{}\ (U \mapsto H^r(U \times_X Y, \mathcal{F}))
\end{align*}
\end{proof}

\begin{cor}
$(R^r\pi_*\mathcal{F})_{\bar{x}} = \varinjlim\limits_{(U,u) \rightarrow \bar{x}} H^r(U \times_X Y, \mathcal{F})$.
\end{cor}

Dacă ne uităm la o varietate $Y$ definită peste un corp separabil închis $k$ (cu morfismul asociat $Y \xrightarrow{\pi}$ Spec $k$), vom constata un lucru interesant. Anume, dat fiind că acoperirile etale ale lui Spec $k$ sunt date de spectrele unor sume directe de $k$-uri, condiția de lipire a secțiunilor ne spune că a da un fascicul pe Spec $k$ este același lucru cu a da un grup abelian, iar prin această echivalență de categorii $\pi_*$ îi corespunde lui $\Gamma$. Rezultă că aici coomologia etală este caz particular de derivare a imaginii directe.

Dacă $\phi:Y \rightarrow X$ este un morfism, putem defini, analog cu teoria clasică, și un functor adjunct lui $\phi_*$, functorul imagine inversă $\phi^* : Ab(X) \rightarrow Ab(Y)$, ce este exact. Rezultă că un șir exact scurt de fascicule etale pe $X$:
$$0 \rightarrow \mathcal{F}' \rightarrow \mathcal{F} \rightarrow \mathcal{F}'' \rightarrow 0$$
induce șirul exact lung format din:
$$\cdots \rightarrow H^r(Y,\phi^*\mathcal{F}') \rightarrow H^r(Y,\phi^*\mathcal{F}) \rightarrow H^r(Y,\phi^*\mathcal{F}'') \rightarrow \cdots$$
iar aplicația naturală $\mathcal{F}(X) \rightarrow \phi^*\mathcal{F}(Y)$ induce, din teoria functorilor derivați, morfisme $H^r(X,\mathcal{F}) \rightarrow H^r(Y,\phi^*\mathcal{F})$ (astfel, coomologia etală este și ea contravariant functorială), ce sunt compatibile cu aplicațiile frontieră din șirurile exacte lungi.

Dacă $j :U \rightarrow X$ este o imersie deschisă, iar $\mathcal{F}$ este un fascicul etal pe $U$, definim prefasciculul $\mathcal{F}_!$ prin:
\begin{align*}
\mathcal{F}_!(g: V \rightarrow X) = \left\{
     \begin{array}{lr}
       \mathcal{F}(V \rightarrow U) & \text{dacă } g(V)\subseteq U\\
       0 & \text{altfel}
     \end{array}
   \right.
\end{align*}
iar fasciculul $j_!\mathcal{F}$ ca fiind fasciculizatul lui $\mathcal{F}_!$.

O aplicație $X \rightarrow S$ se numește {\bf proprie} dacă este separată și pentru orice $T \rightarrow S$, aplicația $X \times_S T \rightarrow T$ este închisă. (O varietate $X$ peste un corp $k$ este {\bf completă} dacă $X \rightarrow$ Spec $k$ este proprie.)

În cazul în care avem, ca mai devreme, $Y \rightarrow$ Spec $k$ o varietate peste un corp separabil închis, teorema de compactificare a lui Nagata ne dă o imersie deschisă $j: Y \rightarrow \bar{Y}$ peste Spec $K$, astfel încât $\bar{Y}$ este completă. Definim $H^r_c(Y,\mathcal{F}) = H^r(\bar{Y},j_!\mathcal{F})$ și o numim {\bf coomologie cu suport propriu}.

Teorema lui Nagata are un enunț ceva mai larg, permițând să plecăm de la morfisme mai generale $Y \rightarrow S$, ceea ce ne duce cu gândul la derivări cu suport propriu ale imaginii directe.

\begin{teo}
Fie $\pi: X \rightarrow S$ proprie, $\mathcal{F}$ fascicul constructibil pe $X$. Atunci $R^r\pi_*\mathcal{F}$ este constructibil pentru orice $r \geq 0$, iar dacă $\bar{x} \rightarrow S$ este un punct geometric, $(R^r\pi_*\mathcal{F})_{\bar{x}} = H^r(X_{\bar{x}},\mathcal{F}_{\mid X_{\bar{x}}})$.
\end{teo}

Dacă luăm pe post de $S$ spectrul unui corp separabil închis, obținem:

\begin{cor}
Grupurile de coomologie ale unui fascicul constructibil peste o varietate completă sunt finite.
\end{cor}

În sfârșit, încheiem acest cerc de idei prin:

\begin{teo}
(Teorema de Schimbare Proprie a Bazei) Fie $\pi: X \rightarrow S$ aplicație proprie, $f: T \rightarrow S$ morfism, $X' = X \times_S T$ și $X\xleftarrow{f'}X'\xrightarrow{\pi'}T$ morfismele de pullback. Dacă $\mathcal{F}$ este un fascicul de torsiune pe $X$ (și deci limită de constructibile), există un izomorfism canonic $f^*(R^r\pi_*\mathcal{F}) \rightarrow R^r\pi'_*(f'^*\mathcal{F})$.
\end{teo}

Să ne întoarcem la ceea ce am vorbit mai devreme despre functorul fibră - după cum am văzut, el induce o echivalență de categorii între acoperirile etale ale unei varietăți conexe și $\pi_1(X,\bar{x})$-mulțimile finite, continue. Aceasta ne spune, în particular, că, dat fiind un grup $G$, a da o acoperire Galois de grup $G$ este același lucru cu a da un morfism continuu de la $\pi_1(X,\bar{x})$ la $G$. Mai departe, o acoperire Galois poate fi văzută ca un spațiu omogen corespunzător fasciculului etal constant de fibră $G$ (notat acum cu $\mathcal{G}$). Aceste spații omogene peste $\mathcal{G}$ sunt structuri asemănătoare fibratelor vectoriale și avem chiar o echivalență de categorii între clasele lor de izomorfism și un anumit grup $H^1$, anume $H^1(X,\mathcal{G})$ (corespunzător lui $H^1(X,\mathcal{O}^*)$ din teoria clasică). Am obținut, care va să zică, relația:
$$H^1(X,\mathcal{G}) \simeq \text{Hom}_{TopGr}(\pi_1(X,\bar{x}),G)$$
care pare că ne va simplifica multe calcule.

Însă, după cum imediat vom vedea, tocmai această relație ne va crea probleme.

\chapter{De la etal la \texorpdfstring{$\ell$}{l}-adic}

Așadar, din raționamentul anterior, avem că pentru orice grup abelian $G$ are loc relația:
$$H^1(X,G)\simeq \text{Hom}_{TopGr}(\pi_1(X,x),G)$$
unde Hom-set-ul din dreapta este în categoria grupurilor topologice (iar $G$ este considerat cu topologia discretă). Dacă vom considera, de pildă, $G$ ca fiind un grup infinit, fără torsiune, spre exemplu $\mathbb{Z}$ (cum se obișnuiește în topologia algebrică), atunci grupul de coomologie va fi trivial. Iată de ce: dacă iau un element al acelui Hom-set, $f: \pi_1(X,x) \rightarrow \mathbb{Z}$ (i.e. un morfism de grupuri continuu), atunci $f^{-1}(0)$ este deschis în $\pi_1(X,x)$. Cum grupul de omotopie este compact, $f^{-1}(0)$ este de indice finit. Prin urmare, $f$ atinge un număr finit de valori, însă singurul subgrup finit al lui $\mathbb{Z}$ este cel trivial. Deci $f$ este morfismul nul, iar $H^1(X,\mathbb{Z})=0$, ceea ce nu corespunde unei oarecari intuiții topologice.

În argumentul de mai sus a intervenit în mod crucial faptul că grupul este luat cu topologia discretă. Dacă am considera un grup topologic cu o structură canonică de grup profinit nediscret, cum ar fi $\mathbb{Z}_\ell$, atunci analogul membrului drept, care este $\text{Hom}_{TopGr}(\pi_1(X,x),\mathbb{Z}_\ell)$, nu va mai fi în mod trivial nul. Însă:
\begin{align*}
\text{Hom}_{TopGr}(\pi_1(X,x),\mathbb{Z}_\ell)&\simeq \text{Hom}_{TopGr}(\pi_1(X,x),\varprojlim_n \frac{\mathbb{Z}}{\ell^n\mathbb{Z}}) \\
&\simeq \varprojlim_n \text{Hom}_{TopGr}(\pi_1(X,x), \frac{\mathbb{Z}}{\ell^n\mathbb{Z}}) \\
&\simeq \varprojlim_n H^1(X, \frac{\mathbb{Z}}{\ell^n\mathbb{Z}}) \\
\end{align*}
Soluția este deci de a defini:
$$H^r(X,\mathbb{Z}_\ell):=\varprojlim_n H^r(X, \frac{\mathbb{Z}}{\ell^n\mathbb{Z}})$$
pentru orice $r$ natural, relație în care se va înțelege că ceea ce este scris în membrul stâng este o notație „pur formală” pentru ceea ce vom numi {\bf coomologie $\ell$-adică}.

Ea se poate defini pe ceea ce se numește un fascicul de $\mathbb{Z}_\ell$-module, adică o familie:
$$\calm=(\{\calm_n\}_n,\{f_{n+1}:\calm_{n+1}\rightarrow\calm\}_n)$$
astfel încât pentru orice $n$, $\calm_n$ este fascicul constructibil de $\frac{\mathbb{Z}}{\ell^n\mathbb{Z}}$-module, $Ker(f_{n+1})=\ell^{n+1}\calm_{n+1}$, $Im(f_{n+1})=\calm_n$.

Un asemenea obiect se numește {\bf plat} dacă pentru orice $n$ și $s$ naturale, șirul:
$$0 \rightarrow \frac{\calm_{n+s}}{\ell^s\calm_{n+s}} \xrightarrow{\cdot\ \ell^n} \calm_{n+s} \rightarrow \frac{\calm_{n+s}}{\ell^n\calm_{n+s}} \rightarrow 0$$
este exact (unde fasiculul din stânga este izomorf cu $\calm_s$, iar cel din dreapta cu $\calm_n$).

Pentru o ultimă dată, ne întoarcem la corespondența monodromică, dar de această dată scrisă pentru fascicule abeliene și chiar $\ell$-adice. Analogul fasiculelor constructibile vor fi familiile de constructibile, pe care le vom numi {\bf fascicule netede}.

\begin{teo}
Functorul fibră induce o echivalență de categorii între $\mathbb{Z}_\ell$-fascicule netede pe o schemă $S$ și modulele libere de rang finit peste $\mathbb{Z}_\ell$ înzestrate cu o acțiune continuă a lui $\pi_1(S,\bar{x})$ ($\mathbb{Z}_\ell$-reprezentările grupului fundamental etal).
\end{teo}

\begin{obs}
Proprietatea rămâne valabilă schimbând $\mathbb{Z}_\ell$ cu $\mathbb{Q}_\ell$ sau chiar cu $\bar{\mathbb{Q}}_\ell$.
\end{obs}

Acum putem defini:
$$H^r(X,\calm):=\varprojlim_n H^r(X,\calm_n)$$
$$H^r_c(X,\calm):=\varprojlim_n H^r_c(X,\calm_n)$$

Aceste grupuri satisfac următoarea proprietate:

\begin{teo}
(Teorema de finitudine) Fie $X$ o varietate definită peste un corp separabil închis $k$. $\calm=\{\calm_n\}_n$ un fascicul plat de $\mathbb{Z}_\ell$-module. Presupunem că $X$ este completă sau $\ell \neq char(k)$. Atunci:
\begin{enumerate}[(i)]
\item modulele $H^r(X,\calm)$ sunt finit generate;
\item pentru orice $n$ există un șir exact lung:
$$\cdots \rightarrow H^r(X,\calm) \xrightarrow{\cdot \ell^n} H^r(X,\calm) \rightarrow H^r(X,\calm_n) \rightarrow H^{r+1}(X,\calm) \rightarrow \cdots$$
\end{enumerate}
\end{teo}

\begin{proof}
Avem pentru orice $n$ și $s$ șirul exact:
$$0 \rightarrow \calm_s \xrightarrow{\cdot\ \ell^n} \calm_{n+s} \rightarrow \calm_n \rightarrow 0$$
compatibil cu familia $\{f_i\}$, care induce șirul exact lung:
$$\cdots H^r(X,\calm_s) \xrightarrow{\cdot \ell^n} H^r(X,\calm_{n+s}) \rightarrow H^r(X,\calm_n) \rightarrow H^{r+1}(X,\calm_s) \rightarrow \cdots$$
(adică al doilea punct al teoremei).
Din compatibilitate, putem trece la limită după $s$ și obținem șirul lung:
$$\cdots H^r(X,\calm) \xrightarrow{\cdots \ell^n} H^r(X,\calm) \rightarrow H^r(X,\calm_n) \rightarrow H^{r+1}(X,\calm)$$
ce induce șirurile exacte scurte de forma:
$$0 \rightarrow \frac{H^r(X,\calm)}{\ell^n H^r(X,\calm)} \rightarrow H^r(X,\calm_n) \rightarrow H^{r+1}(X,\calm)_{\ell^n} \rightarrow 0$$

Trecem la limită după $n$ și obținem:
$$0 \rightarrow \varprojlim_n \frac{H^r(X,\calm)}{\ell^n H^r(X,\calm)} \rightarrow H^r(X,\calm) \rightarrow \varprojlim_n H^{r+1}(X,\calm)_{\ell^n} \rightarrow 0$$

Însă
\begin{align*}
\varprojlim_n H^{r+1}(X,\calm)_{\ell^n} &= \varprojlim_n (H^{r+1}(X,\calm)_{\ell^n} \xleftarrow{\ell} H^{r+1}(X,\calm)_{\ell^{n+1}})\\
&=\{(x_n)_n \mid \forall n, \ell^n x_n=0, \ell x_{n+1}=x_n\}\\
&=0
\end{align*}
(fiindcă nu există elemente nenule infinit divizibile, $H^{r+1}(X,\calm)$ fiind limită proiectivă de grupuri finite de $\ell^n$-torsiune). Deci $H^r(X,\calm)\simeq \varprojlim_n \frac{H^r(X,\calm)}{\ell^n H^r(X,\calm)}$, ce este finit generat.
\end{proof}

Vom nota în continuare, la fel de formal,
$$H^r(X,\Qell) = H^r(X,\mathbb{Z}_\ell) \otimes_{\mathbb{Z}_\ell} \Qell$$

Un rezultat ce se moștenește în mod trivial la această coomologie $\ell$-adică este dimensiunea coomologică.

\begin{teo}
(Formula Künneth) Are loc izomorfismul de inele graduate:
$$H^*(X\times Y,\Qell) \simeq H^*(X,\Qell) \otimes_{\Qell} H^*(Y,\Qell)$$
\end{teo}

În continuare, vom indica fundamentele teoriei intersecției subvarietăților și legătura cu instrumentul coomologic.

Dacă $X$ este o varietate nesingulară, numim un {\bf ciclu prim} o subvarietate închisă ireductibilă. Notăm cu $C^r(X)$ grupul abelian liber generat de ciclii primi de codimensiune $r$, iar cu $C^*(X)$ suma directă a acestor $C^r$-uri.

Numim doi cicli primi {\bf rațional echivalenți} dacă există o deformare plată între cele două subvarietăți, cu spațiul de bază să fie dreapta proiectivă, și care să fie inclusă în produsul dreptei proiective cu varietatea ambiantă.

Dacă factorizăm fiecare $C^r$ la relația de congruență indusă de echivalența rațională, vom obține grupuri notate cu $CH^r(X)$. Grupul $CH^*(X)$, suma directă a lor, are în mod natural o structură de inel graduat (în care punem gradul elementelor din $C^r$ să fie $2r$, pentru a se potrivi mai jos cu coomologia), numit {\bf inelul Chow}, cu următoarele proprietăți:
\begin{itemize}
\item $CH^*(X)$ este comutativ;
\item asocierea este contravariant-functorială, în sensul că dacă avem $f: X \rightarrow X'$ morfism de varietăți, funcția asociată $f^*:CH(X') \rightarrow CH(X)$ este morfism de inele și $f^*g^* = (g \circ f)^*$;
\item pntru morfismele proprii $f: X \rightarrow X'$, există și un push-forward $f_*:CH(X) \rightarrow CH(X')$ ce este morfism de grupuri graduate (cu o translație în grade), la fel oarecum functorial;
\item dacă $f: X \rightarrow X'$ e propriu, atunci $f_*(x \cdot f^*(y)) = f_*(x) \cdot y$;
\item dacă $Y$ și $Z$ sunt cicli pe $X$ și $\Delta: X \rightarrow X \times X$ este morfismul diagonal, atunci $Y \cdot Z = \Delta^*(Y \times Z)$;
\item dacă $Y$ și $Z$ sunt subvarietăți ce se intersectează propriu (i.e. orice componentă ireductibilă $W$ a lui $Y \cap Z$ are codimensiunea în $X$ egală cu suma codimensiunilor lui $Y$ și $Z$), atunci:
$$Y \cdot Z = \sum \iota(Y,Z;W_j) W_j$$
\item dacă $Y$ este subvarietate a lui $X$, iar $Z$ un divizor efectiv pe $X$ ce intersectează propriu pe $Y$, atunci $Y \cdot Z$ este ciclul asociat divizorului $Y \cap Z$ pe $Y$.
\end{itemize}

Ideea fundamentală aici este existența unei transformări (naturale în variabila $X$):
$$cl_X : CH^*(X) \rightarrow H^*(X,\Qell)$$
pentru orice $\ell$ diferit de caracteristica corpului peste care lucrăm (fiind forțați de șirul exact Kummer). Aceste morfisme sunt morfisme de inele graduate. În acest mod, putem reduce calculele de intersecții la calcule de produse {\it cup}.

Precum formula Künneth, și dualitatea Poincaré este un rezultat ce are loc și în context $\ell$-adic. Însă, după cum știm, dualitatea Poincaré reprezintă un fenomen strâns legat de orientare. Geometria pe care o dezvoltăm aici își trage intuiția mai ales din geometria complexă, unde orientarea este indicată de alegerea unei rădăcini a lui $-1$. Grupul de coeficienți cu care lucrăm noi este $\mathbb{Q}_\ell$, obținut din $\mathbb{Z}_\ell$, care la rându-i este limită proiectivă de $\mathbb{Z}/\ell^n\mathbb{Z}$-uri. Fiecare $\mathbb{Z}/\ell^n\mathbb{Z}$ este izomorf necanonic cu grupul rădăcinilor de ordin $\ell^n$ ale unității (notat cu $\mu_{\ell^n}(k)$, unde $k$ este corpul peste care lucrăm), acesta din urmă fiind varianta „neorientată” a lui $\mathbb{Z}/\ell^n\mathbb{Z}$. Din el putem construi $\mathbb{Z}_\ell(1)=\varprojlim \mu_{\ell^n}$ și apoi $\mathbb{Z}_\ell(n)=(\mathbb{Z}_\ell(1))^{\otimes_{\mathbb{Z}_\ell}^n}$. Și, firește, $\mathbb{Q}_\ell(n)$ este $\mathbb{Z}_\ell(n) \otimes_{\mathbb{Z}_\ell} \mathbb{Q}_\ell$. Acum putem da versiunea $\ell$-adică a dualității Poincaré fără a comite alegeri necanonice.

\begin{teo}
(Dualitatea Poincaré) Fie $X$ o varietate nesingulară de dimensiune $d$ definită peste un corp $k$ algebric închis. Atunci:
\begin{enumerate}[a)]
\item există și este unică o aplicație (ce este în particular izomorfism):
$$\eta_X : H^{2d}_c(X,\Qell) \rightarrow \Qell$$
(numită {\bf aplicația urmă}) astfel încât pentru orice $P$ punct închis, $\eta_X(cl_X(P))=1$;
\item pentru orice $r$ există o biliniară nedegenerată:
$$H^r_c(X,\Qell) \times H^{2d-r}(X,\Qell(-d)) \rightarrow H^{2d}_c(X,\Qell(d)) \simeq \Qell$$
\end{enumerate}
Iar pentru varietățile complete, putem elimina $c$-urile și scrie simplu biliniara ca:
$$H^r(X,\Qell) \times H^{2d-r}(X,\Qell) \rightarrow H^{2d}(X,\Qell) \simeq \Qell(-d)$$
\end{teo}

Fie $\pi : Y \rightarrow X$ o aplicație proprie între varietăți complete definite peste un corp $k$ algebric închis. Notăm cu $a$ dimensiunea lui $X$, cu $d$ dimensiunea lui $Y$ și cu $c$ numărul $d-a$.

Această aplicație induce în coomologie:
$$\pi^*: H^{2d-r}(Y,\Qell(d)) \rightarrow H^{2d-r}(Y,\Qell(d))$$
de unde obținem via dualitatea Poincaré adjuncta ei:
$$\pi_* : H^r(Y,\Qell) \rightarrow H^{r-2c}(X,\Qell(-c))$$
ce satisface, firește proprietatea definitorie:
$$\langle \pi_*(y),x \rangle=\langle y,\pi^*(x)\rangle$$
pentru orice $x$, $y$ potrivite.

Adjuncta, numită {\bf aplicația Gysin}, satisface un număr de proprietăți.

\begin{propr}
Dacă $\pi : Y \rightarrow Z$ este imersie închisă, atunci $\pi_*(1_Y)=cl_Z(Y)$.
\end{propr}

\begin{propr}
Pentru orice $y$ din $H^{2d}(Y,\Qell)$, $\eta_X(\pi_*(y))=\eta_Y(y)$.
\end{propr}

\begin{proof}
\begin{align*}
\eta_X(\pi_*(y))&=\eta_X(\pi_*(y)\smile 1_X)\\
&=\langle \pi_*(y), 1_X \rangle \\
&= \langle y, \pi^*(1_X) \\
&= \langle y, 1_Y \rangle \\
&= \eta_Y (y \smile 1_Y) \\
&= \eta_Y(y).
\end{align*}
\end{proof}

\begin{propr}
Pentru orice $x$, $y$ adecvate, $\pi_*(y \smile \pi^*(x)) = \pi_*(y) \smile x$ (asemănător cu proprietatea de la inelul Chow).
\end{propr}

\begin{proof}
$\langle \pi_*(y \smile \pi^*(x)), x' \rangle = \langle y \smile \pi^*(x), \pi^*(x') \rangle = \langle y, \pi^*(x \smile x') \rangle = \langle \pi_*(y),x\smile x' \rangle = \langle \pi_*(y) \smile x, x' \rangle$.
\end{proof}

\begin{propr}
Dacă $\pi:Y \rightarrow X$ este în plus finită de grad $\delta$ (implicit varietățile au aceeași dimensiune) atunci, $\pi_* \circ \pi^*$ este înmulțirea cu $\delta$.
\end{propr}

\begin{proof}
În primul rând $\langle \pi_*(1),x\rangle = \langle1,\pi^*(x)\rangle = \langle1,\delta x\rangle = \langle \delta,x\rangle$, astfel că $\pi_*(1)=\delta$.
Deci $\pi_*(\pi^*(x)) = \pi_*(1 \smile \pi^*(x)) = \pi_*(1) \smile x = \delta \smile x = \delta x$.
\end{proof}
Rezultă deci că în acest ultim caz $\pi_*$ este identitatea pe $H^{2d}(X,\Qell)$ și înmulțirea cu $\delta$ pe $H^0(X,\Qell)$.

O consecință mai importantă a dualității Poincaré este:

\begin{teo}
(Teorema slabă Lefschetz) Fie $X$ o varietate proiectivă de dimensiune $d$ și $H$ un hiperplan în spațiul ambiant astfel încât complementara în $X$ a lui $A = X \cap H$ este netedă. Atunci aplicațiile canonice:
$$H^r(X,\mathbb{Q}_\ell) \rightarrow H^r(A,\mathbb{Q}_\ell)$$
sunt bijective pentru $r < d-1$ și injective pentru $r=d-1$.
\end{teo}

Ultimul rezultat ce mai trebuie arătat pentru a finaliza mașinăria coomologică este formula de punct fix a lui Lefschetz. Ceea ce este interesant este că ea decurge formal din teoremele expuse mai devreme.

Presupunem că $X$ este o varietate completă nesingulară peste un corp $k$, iar $\phi$ o aplicație regulată de la $X$ la $X$. Atunci $\Gamma_\phi \cdot \Delta_X$ (ca intersecție a doi cicli în $X \times X$) contabilizează punctele fixe ale lui $\phi$, numărate cu multiplicități. Are loc:

\begin{teo}
(Formula de punct fix Lefschetz)
$$\langle \Gamma_\phi \cdot \Delta_X \rangle = \sum\limits_r (-1)^r Tr(\phi^*_{\mid H^r(X,\Qell)})$$
\end{teo}

\begin{proof}
În primul rând, știm că dacă $\{e_i\}_i$ este o bază într-un spațiu vectorial finit-dimensional $V$ și $\phi$ un endomorfism al lui $V$ avem relația:
$$Tr(\phi)=\sum\limits_i e_i^*(\phi(e_i))$$
unde prin $\{e_i^*\}_i$ am notat baza duală corespunzătoare.

Peste tot vom identifica (din Künneth) $H^*(X\times X,\Qell)$ cu $H^*(X,\Qell) \otimes_{\Qell} H^*(X,\Qell)$.

Fie, pentru orice $r$ $\{e_i^r\}_i$ bază pentru $H^r(X,\Qell)$ (cu $e^{2d}_1=cl_X(P)$ și, firește, $e^{2d}_1 \otimes e^{2d}_1 = cl_{X\times X}(P)$) și $\{f_i^{2d-r}\}_i$ bază pentru $H^{2d-r}(X,\Qell)$ duală relativ la produsul {\it cup}, i.e. pentru orice $i,j,r$, avem $e_i^r \smile f_j^{2d-r} = \delta_{ij}e^{2d}_1$.

Scriem $cl(\Gamma_\phi)=\sum\limits_{r,i} a_i^r \otimes f_i^{2d-r}$. Vrem să aflăm coeficienții $a_i^r$. Pentru aceasta, notând cele două proiecții $X \times X \rightarrow X$ cu $p$ și $q$, avem:
\begin{align*}
a^r_j&=p_*(a^r_j \otimes e^{2d}_1)\\
&=p_* ((\sum\limits_i (a^r_i \otimes f^{2d-r}_i)) \smile (1 \otimes e^r_j)) \\
&= p_*(cl(\Gamma_\phi) \smile q^* e^r_j) \\
&= p_*( (1,\phi)_*(1) \smile q^* e^r_j) \\
&= p_* (1,\phi)_* (1 \smile (q \circ (1,\phi))^*e_j^r)\\
&= id_*(1 \smile \phi^*(e_j^r))\\
&= \phi^*(e_j^r)
\end{align*}

Deci $cl(\Gamma_\phi)=\sum\limits_{r,i} \phi^*(e^r_i) \otimes f^{2d-r}_i$ și:
$$cl(\Delta_X)=cl(\Gamma_{id_X})=\sum\limits_{r,i} e^r_i \otimes f^{2d-r}_i = \sum\limits_{r,i} (-1)^r f_i^{2d-r} \otimes e^r_i$$
Ca urmare:
\begin{align*}
cl(\Gamma_\phi \cdot \Delta_X) &= cl(\Gamma_\phi) \smile cl(\Delta_X) \\
&= \sum\limits_{r,i} ((-1)^r \phi^*(e^r_i) \smile f^{2d-r}_i) \otimes (f^{2d-r}_i \smile e^r_i) \\
&= \sum\limits_r (-1)^r Tr(\phi^*_{\mid H^r(X,\Qell)}) (e^{2d}_1 \otimes e^{2d}_1)
\end{align*}
de unde rezultă imediat formula.
\end{proof}

Ca atare, putem număra puncte.

\chapter{Numărarea punctelor}

Problema pe care urmează să o formulăm a pornit de la cea a studiului modului cum variază numărul de puncte de pe o curbă eliptică, în momentul în care extindem corpul de bază. Ne este cunoscută din studiul acelor curbe inegalitatea Hasse-Weil, care spune că pentru orice curbă eliptică $X$ definită peste un corp finit $\mathbb{F}_q$, dacă notăm $N_m(X)=\#X(\mathbb{F}_{q^m})$, are loc relația:
$$\mid N_m(X) - (q^m+1)\mid\  \leq 2\sqrt{q^m}$$

Mai precis, există două numere algebrice $\alpha_1,\alpha_2$ de modul $\sqrt{q}$ astfel încât pentru orice $m$:
$$N_m(X) = 1 - \alpha_1^m - \alpha_2^m + q^m$$

André Weil a propus următoarea generalizare:
\begin{teo}
(Conjecturile Weil) Fie $X$ o varietate proiectivă netedă definită peste $\mathbb{F}_q$ de dimensiune $d$. Atunci:
\begin{enumerate}
\item există $2d$ numere naturale $b_0,...,b_{2d}$ și o familie de numere complexe $\{a_{j,u}\}_{j\in\overline{0,2d}\atop s\in\overline{1,b_j}}$ astfel încât pentru orice $m$ am:
$$N_m(X)=\sum\limits_{j=0}^{2d} (-1)^j (\sum\limits_{u=1}^{b_j} \alpha_{j,u}^m)$$
Mai mult, $b_0=b_{2d}=1, \alpha_{0,1}=1, \alpha_{2d,1}=q^d$. Numărul $\sum\limits_j (-1)^j b_j$ va fi notat cu $\chi$.
\item pentru orice $j$, $b_j$ este egal cu $b_{2d-j}$, iar $(\frac{q^d}{\alpha_{2d-j,1}},...,\frac{q^d}{\alpha_{2d-j,b_j}})$ e o permutare a enumerării $(\alpha_{j,1},...,\alpha_{j,b_j})$.
\item pentru orice $j,u$, avem că $\alpha_{j,u}$ e număr algebric de modul $q^{\frac{j}{2}}$.
\end{enumerate}
\end{teo}

Se observă că dacă X este curbă eliptică se reconstituie relația de mai devreme, cu $b_1=2=dim\ H^1(\mathbb{C}/\Lambda, \mathbb{Q})$ (pentru $\Lambda$ o latice în planul complex).

Un mod mai pragmatic de a exprima conjecturile Weil este reprezentat de instrumentul funcțiilor generatoare.

Ne bazăm pe identitatea formală:
$$log(\frac{1}{1-x})=\sum\limits_{m=1}^{\infty} \frac{x^m}{m}$$
scrisă eventual
$$\frac{1}{1-x}=exp(\sum\limits_{m=1}^{\infty} \frac{x^m}{m})$$
ce se poate verifica via expansiune în serie Taylor în jurul lui zero.

Definind $Z_X(t) = exp(\sum\limits_{m=1}^{\infty} N_m(X) \frac{t^m}{m})$, obținem din punctul 1 al conjecturilor:
\begin{align*}
Z_X(t)&=exp(\sum\limits_{m=1}^{\infty} \sum\limits_{j=0}^{2d} (-1)^j \sum\limits_{u=1}^{b_j} \alpha_{j,u}^m \frac{t^m}{m})\\
&=\prod\limits_{j=0}^{2d} (\prod_{u=1}^{b_j} exp(\sum\limits_{m=1}^{\infty} \frac{(a_{j,u} t)^m}{m}))^{(-1)^j}\\
&=\prod\limits_{j=0}^{2d} (\frac{1}{\prod\limits_{u=1}^{b_j}(1-\alpha_{j,u}t)})^{(-1)^j}\\
&=\prod\limits_{j=0}^{2d} P_j(t)^{(-1)^{j+1}}
\end{align*}
unde am notat $P_j(t)=\prod\limits_{u=1}^{b_j}(1-\alpha_{j,u}t)$ (și am $P_0(t)=1-t$, $P_{2d}(t)=1-q^dt$).

Vom deriva acum din punctul 2 o relație pe care o va satisface $Z_X(t)$. Aplicăm relația de permutare între enumerări și obținem:
\begin{align*}
P_{2d-j}(t)&=\prod_u(1-\alpha_{2d-j,u}t)=\prod_u(1-\frac{q^d}{a_{j,u}}t)\\
&=(\prod_u \alpha_{j,u})^{-1} \prod_u(\alpha_{j,u}-q^dt)\\
&=(\prod_u \alpha_{j,u})^{-1} (-1)^{b_j} (q^dt)^{b_j} \prod_u (1-\frac{\alpha_{j,u}}{q^dt})\\
&=(\prod_u \alpha_{j,u})^{-1} (-1)^{b_j} (q^dt)^{b_j}P_j(\frac{1}{q^dt})
\end{align*}

Folosim acum atât simetria $b_j$-urilor cât și permutarea enumerărilor:
\begin{align*}
P_j(t)P_{2d-j}(t)&=(q^dt)^{2b_j} (q^d)^{-b_j} P_j(\frac{1}{q^dt}) P_{2d-j}(\frac{1}{q^dt})\\
&=(q^d)^{\frac{b_j+b_{2d-j}}{2}} t^{b_j+b_{2d-j}}P_j(\frac{1}{q^dt}) P_{2d-j}(\frac{1}{q^dt})\\
\end{align*}

Însă $(\prod\limits_s \alpha_{d,u})^2=(q^d)^{b_d}$, deci $\prod\limits_s \alpha_{d,u}=\pm (q^d)^{\frac{b_d}{2}}$, și pot scoate relația (pentru indicele $d$):
$$P_d(t)=\pm (-1)^{b_d} (q^dt)^{b_d} (q^d)^{\frac{b_d}{2}} P_d(\frac{1}{q^dt})$$

Și obțin astfel formula pentru funcția $Z_X$:
\begin{align*}
Z_X(t)&=\prod_{j=0}^{2d} P_j(t)^{(-1)^{j+1}}\\
&=\pm \prod_{j=0}^{2d} P_j(\frac{1}{q^dt})^{(-1)^{j+1}} (q^d)^{-\frac{\sum\limits_j (-1)^{j+1} b_j}{2}} t^{-\sum\limits_j (-1)^{j+1} b_j}\\
&=\pm q^{\frac{d\chi}{2}} t^{\chi} Z_X(\frac{1}{q^dt})
\end{align*}
numită {\bf ecuația funcțională} a lui $Z_X$.

O altă reformulare ne este dată de următoarea substituție:
$$\zeta_X(s)=Z_X(q^{-s})$$

Notând pentru un punct închis $x \in X$ cu $\kappa(x)$ corpul rezidual al său, cu $deg(x)$ gradul extinderii $\kappa(x) : \mathbb{F}_q$ și cu
\begin{align*}
N_m(x) = \left\{
     \begin{array}{lr}
       deg(x) & \text{dacă } deg(x)\ \mid \ m\\
       0 & \text{altfel}
     \end{array}
   \right.
\end{align*}
avem din cele cunoscute de la teoria schemelor:
$$N_m(X)=\sum\limits_x N_m(x)$$

Obținem rescrierile:
\begin{align*}
Z_X(t)&=exp(\sum_{m\geq 1} N_m(X) \frac{t^m}{m}) = exp(\sum_{m\geq 1} \sum\limits_x N_m(x) \frac{t^m}{m})\\
&=exp(\sum_x \sum_{m \geq 1 \atop deg(x) \mid  m} N_m(x) \frac{t^m}{m})\\
&=exp(\sum_x \sum_{n \geq 1} deg(x) \frac{t^{n\cdot deg(x)}}{n \cdot deg(x)})\\
&=exp(\sum_x \sum_{n\geq 1} \frac{(t^{deg(x)})^n}{n})=exp(\sum_x log \frac{1}{1-t^{deg(x)}})\\
&=exp\ log \prod_x \frac{1}{1-t^{deg(x)}} = \prod_x \frac{1}{1-t^{deg(x)}}
\end{align*}
și deci
$$\zeta_X(s)=\prod_x \frac{1}{1-(q^{deg(x)})^{-s}} = \prod_x \frac{1}{1-(\#\kappa(x))^{-s}}$$
Ultima formulă are sens pentru o schemă oarecare, nu neapărat peste un corp finit. De pildă, înlocuind $X$ cu $\text{Spec }\mathbb{Z}$, apare:
$$\zeta_{\text{Spec }\mathbb{Z}}(s)=\prod_{\mathfrak{p} \in \text{Max }\mathbb{Z}} \frac{1}{1-(\#(\frac{\mathbb{Z}}{\mathfrak{p}}))^{-s}} = \prod_{p\text{ prim}} \frac{1}{1-p^{-s}}$$
binecunoscuta funcție zeta a lui Riemann (punctele închise din $\text{Spec }\mathbb{Z}$ sunt precis idealele maximale ale lui $\mathbb{Z}$), ceea ce ne justifică notația.

Atenție, însă: funcția $Z_X$ nu are sens decât pentru scheme definite peste un corp finit!

Este clar că $s\in\mathbb{C}$ este zerou, respectiv pol, pentru $\zeta_X$ dacă și numai dacă $q^{-s}$ va avea aceeași calitate pentru $Z_X$ (și orice $t \in \mathbb{C}^*$ se poate scrie ca $q^{-s}$, e drept, într-o infinitate de moduri). Iau un astfel de $s$. Din exprimarea rațională a lui $Z_X$ rezultă că $q^{-s}=\frac{1}{\alpha_{j,u}}$ pentru anumite $j$ și $u$, iar ultima afirmație din conjecturi implică $\mid q^{-s}\mid =q^{-\frac{j}{2}}$. Scriu $s=a+bi$, cu $a,b\in\mathbb{R}$, și am $q^{-s}=e^{-s\ ln q} = e^{-a\ ln q- bi\ ln q}$. Deci $q^{-\frac{j}{2}}=\mid e^{-a\ ln q- bi\ ln q}\mid =q^{-a}$ și $\frac{j}{2}=a=\mathbb{R}e\ s$.

Invers, acum, dacă fac presupunerea că orice $s$ zerou sau pol al lui $\zeta_X$ are proprietatea că $\mathbb{R}e\ s=\frac{j}{2}$ (cu $j\in\overline{0,2d}$), pot face următorul raționament. Iau $t \in \mathbb{C}^*$ zerou sau pol pentru $Z_X$ și fie $s$ cu $t=q^{-s}$. Atunci $s$ este zerou/pol pentru $\zeta_X$ și $\mathbb{R}e\ s=\frac{j}{2}$. Ca înainte, avem $\mid t\mid =\mid q^{-s}\mid =q^{-\mathbb{R}e\ s}=q^{-\frac{j}{2}}$. Deci zerourile și polii lui $Z_X$ au modulul $q^{-\frac{j}{2}}$.

Rezultă că acea condiție 3 este echivalentă cu faptul că funcția zeta a varietății are zerourile și polii pe liniile $\mathbb{R}e\ s=\frac{j}{2}$, ceea ce justifică numele dat condiției de {\bf ipoteză Riemann}.

Mai mult decât atât, această separare după modul a zerourilor și polilor lui $Z_X$ ne permitm să extragem $P_j$-urile din $Z_X$ și să obținem în mod reciproc din ecuația funcțională condiția 2 din conjecturile Weil.

Toate aceste manipulări de formule ne învață să apreciem modul cum coomologia etală conduce în mod natural măcar la expresia rațională și la ecuația funcțională a funcției $Z_X$.

Să vedem cum. Mai întâi, vom obține o altă descriere a lui $X(\mathbb{F}_{q^m})$ cu ajutorul aplicației Frobenius.

Orice $\mathbb{F}_q$-algebră $A$ admite endomorfismul Frobenius $a \mapsto a^q$, ce se dualizează la o aplicație $\text{Spec } A \rightarrow \text{Spec } A$. Această familie de aplicații se prelungește unic la o întreagă transformare naturală $\{F_X : X \rightarrow X\}_{X \in \mathbb{F}_q-Sch}$. Din naturalitate, rezultă că acționează pe varietăți afine sau proiective în modul firesc, prin ridicarea la puterea $q$ a coordonatelor. În particular $F_X$ are gradul $q^{dim X}$, după cum se verifică ușor pe spațiile afine.

Rezultă deci  $X(\mathbb{F}_{q^m})=Fix(F_X^m)$.

\begin{lema}
$\Gamma_{F^m} \cdot \Delta_X = \sum\limits_{P \in Fix(F_X^m)} P$ (în sensul că toate apar cu multiplicitate $1$).
\end{lema}

\begin{proof}
E suficient să arătăm pentru $m=1$ ($F^m$ este Frobeniusul lui $\mathbb{F}_{q^m}$). Notez $F=F_X$.

Fie $P\in Fix(F)$. Înlocuiesc $X$ cu o vecinătate afină a lui $P$, să zicem $U=\text{Spec }A$, cu $A=\mathbb{F}_q[t_1,...,t_n]=\frac{\mathbb{F}_q[T_1,...,T_n]}{\mathfrak{a}}$.

Atunci pentru orice $i$ am $t_i \circ F = t_i^q$ și $(dt_i)_P \circ (dF)_P = (dt_i^q)_P = qt_i^{q-1}(dt_i)_P = 0$.

Ca urmare diferențiala lui $F$ în $P$ este zero, ca urmare graficul lui Frobenius nu este tangent la diagonală în $(P,P)$, iar numărul de intersecție (multiplicitatea) este $1$. 
\end{proof}

Aplicând formula de punct fix a lui Lefschetz, rezultă (pentru $(\ell,q)=1$):
$$N_m(X) = \sum\limits_r (-1)^r Tr(F^m_{\mid  H^r(X,\Qell)})$$

Putem folosi formula pentru a prelucra funcția $Z$ a lui $X$:
\begin{align*}
Z_X(t)&=exp(\sum\limits_{m=1}^{\infty} N_m(X) \frac{t^m}{m})\\
&=exp(\sum\limits_{m=1}^{\infty} (\sum\limits_{r=0}^{2d} (-1)^r Tr(F^m_{\mid  H^r(X,\Qell)}) \frac{t^m}{m})\\
&=\prod\limits_{r=0}^{2d}exp(\sum\limits_{m=1}^{\infty}Tr(F^m_{\mid  H^r(X,\Qell)})  \frac{t^m}{m}) ^ { (-1)^r }
\end{align*}

Scriem acum fiecare $F_{\mid  H^r(X,\Qell)}$ ca matrice pătratică superior triunghiulară (eventual peste o extindere a lui $\mathbb{F}_q$) cu numărul de linii egal cu $b_r=dim\ H^r(X,\Qell)$. Se observă atunci că dacă elementele de pe diagonală (valorile proprii) sunt $\alpha_{r,1},...,\alpha_{r,b_r}$, atunci $Tr(F^m_{\mid  H^r(X,\Qell)})$ va fi egal cu $\sum\limits_{i=1}^{b_r} \alpha_{r,i}^m$ și deci pot aplica exact același raționament ca mai devreme pentru a exprima rațional pe $Z_X$.

Vom avea $Z_X \in \Qell(t) \cap \mathbb{Q}[[t]] \subseteq \mathbb{Q}(t)$, însă aceasta nu ne garantează că fiecare $P_j$ este în $\mathbb{Q}[t]$ (altfel spus, că este „independent de $\ell$”) - aceasta se poate face doar presupunând ipoteza lui Riemann, care ne permite, după cum am spus și mai devreme, să separăm $P_j$-urile după modulul rădăcinilor.

Altă consecință a raționamentului precedent a fost că am identificat $\alpha_{j,u}$-urile ca fiind valorile proprii ale operatorilor induși de Frobenius pe spațiile de coomologie.

Aceasta ne arată în particular că $b_0=b_{2d}=1$.

Trecem acum la demonstrarea simetriei între valorile proprii pe spațiile de ordin $r$ și $2d-r$, relație care după cum am observat ne implică o ecuație funcțională.

Avem biliniara nedegenerată dată de dualitatea Poincaré:
$$\langle,\rangle=\eta_X\ \circ \smile\ : H^{2d-r}(X,\Qell) \times H^{r}(X,\Qell) \rightarrow H^{2d}(X,\Qell) \simeq \Qell(-d)$$
ce din start ne indică $b_r=b_{2d-r}$.

Știm că $F^*_r=F_{\mid  H^r(X,\Qell)}$ are adjunct relativ la $\langle,\rangle$:
$$\langle F_{*2d-r}(x),x'\rangle=\langle x,F^*_r(x')\rangle, \forall\ x \in H^{2d-r}(X,\Qell), x' \in H^{r}(X,\Qell)$$

Din considerente de algebră liniară rezultă că valorile proprii ale lui $F^*_r$ coincid cu ale lui $F_{*2d-r}$. Însă $F_{*r} \circ F^*_r = q^d (=deg\ F)$.

De aici rezultă că dacă $(\alpha_1,...,\alpha_v)$ sunt valorile proprii ale lui $F^*_r$, $(\frac{q^d}{\alpha_1},...,\frac{q^d}{\alpha_v})$ sunt cele ale lui $F_{*r}$, deci (din cele anterioare) și ale lui $F^*_{2d-r}$, i.e. exact ce ni se cerea. Iar faptul că $\alpha_{0,1}=1$ și $\alpha_{2d,1}=q^d$ rezultă din modul cum acționează operatorii Frobenius pe $H^0$, respectiv pe $H^{2d}$.

În acest moment, tot ce ne rămâne este să demonstrăm ipoteza lui Riemann.

\chapter{«La conjecture de Weil»}

Mai precis, ce avem de demonstrat este:

\begin{teo}
(Deligne, 1974) Fie $X$ o varietate proiectivă $d$-dimensională absolut nesingulară și absolut ireductibilă definită peste $\mathbb{F}_q$; $\alpha$ o valoare proprie a lui $F^*_r$; $\tau$ o scufundare a lui $\Qell$ în $\mathbb{C}$.

Atunci $\tau(\alpha)$ (mai departe simbolul $\tau$ va fi subînțeles) este algebric și de modul $q^{\frac{r}{2}}$.
\end{teo}

În primul rând, se observă că este suficient să demonstrăm pentru varietatea obținută după o schimbare de bază spre o extindere finită a corpului de definiție - să zicem, de grad $m$. Asta deoarece operatorul Frobenius pe varietatea nouă va fi puterea $m$ a celui de pe varietatea veche. Dacă $\alpha$ este valoare proprie pentru Frobenius-ul vechi, $\alpha^m$ este pentru cel nou. Din ipoteza noastră, $\alpha^m$ are modulul $(q^{m})^{\frac{r}{2}}$. Ca urmare, $\alpha$ va avea modulul $q^{\frac{r}{2}}$. Aceasta ne va permite să facem un număr finit de extinderi finite de-a lungul demonstrației, fără a pierde din generalitatea enunțului.

În al doilea rând, se vede că pot demonstra doar pentru spațiile cu rangul cel mult $d$. Aceasta deoarece de la rangul $d+1$ încolo, valorile proprii au simetria implicată de dualitatea Poincaré pe care am văzut-o mai devreme.

În al treilea rând, vom arăta că este suficent să arătăm teorema doar pentru $r=d$. Iată de ce: din teorema lui Bertini, există $Z \subset X$ o secțiune hiperplană netedă (eventual extinzând corpul).  Aplicând teorema Lefschetz slabă, aplicația canonică (ce este compatibilă cu Frobenius) $H^r(X,\Qell) \rightarrow H^r(Z,\Qell)$ este injectivă pentru $r \leq d - 1$ și pot aplica un raționament prin inducție (pasul de bază fiind varietățile zero-dimensionale, pentru care clar este adevărat, din modul cum acționează Frobenius pe $H^0$).

În al patrulea rând, putem să arătăm chiar și numai pentru varietățile de dimensiune pară, iar pentru acelea, doar că valorile proprii $\alpha$ corespunzătoare spațiului de coomologie din mijloc satisfac inegalitatea
$$q^{\frac{d}{2}-\frac{1}{2}}\leq\mid \alpha\mid \leq q^{\frac{d}{2}+\frac{1}{2}}$$
Presupunând că am arătat așa ceva, vreau să demonstrez teorema pentru o varietate oarecare $X$ și $\alpha$ valoare proprie a lui $F^*_d$. Fie $k$ număr natural. Iau $Y$ ca fiind produsul lui $X$ cu el însuși de $2k$ ori. Conform Künneth, $H^d(X,\Qell)^{\otimes 2k}$ se scufundă în $H^{2kd}(Y,\Qell)$, iar $\alpha^{2k}$ va fi valoare proprie a lui Frobenius aplicat pe spațiul de pe urmă. Ca urmare, va avea loc:
$$q^{\frac{2kd}{2}-\frac{1}{2}}\leq\mid \alpha\mid ^{2k}\leq q^{\frac{2kd}{2}+\frac{1}{2}}$$
Scoțând radical de ordin $2k$, obțin:
$$q^{\frac{d}{2}-\frac{1}{4k}}\leq\mid \alpha\mid \leq q^{\frac{d}{2}+\frac{1}{4k}}$$
Cum relația are loc pentru $k$ arbitrar, trecându-l la infinit, obțin $\mid \alpha\mid =q^{\frac{d}{2}}$.

De acum înainte, prin urmare, $d$ va fi par.

În acest moment, după ce am încheiat reducerile geometrice de mai devreme, putem trece la miezul problemei și să introducem tehnica numită {\it pencil Lefschetz}.

Alegem o scufundare a lui $X$ într-un $\mathbb{P}^N$ și luăm $L$ un subspațiu liniar proiectiv de codimensiune $2$ ce intersectează transversal pe $X$. Mulțimea hiperplanelor din $\mathbb{P}^N$ ce conțin pe $L$ are ca spațiu de moduli pe $\mathbb{P}^1$ și deci poate fi organizată ca o familie $\{H_d\}_{d\in\mathbb{P}^1}$. Iau apoi mulțimea:
$$\widetilde{X}=\{(x,d)\in X \times \mathbb{P}^1 \mid x \in H_d\}$$
ce are structură de varietate algebrică, anume este {\it eclatarea lui $X$ în $L\cap X$} (este netedă, din faptul că $L$ intersectează transversal pe $X$). Ea este înzestrată cu două aplicații canonice:
$$X \leftarrow \widetilde{X} \xrightarrow{f} \mathbb{P}^1$$
iar din teoria eclatării, aplicația $H^d(X,\Qell) \rightarrow H^d(\widetilde{X},\Qell)$ este injectivă. Putem trece deci de la $X$ la $\widetilde{X}$ fără probleme.

Ceea ce caracterizează $\widetilde{X}$ este că este înzestrat cu aplicația $f : \widetilde{X} \rightarrow \mathbb{P}^1$. Ea se numește {\bf pencil Lefschetz} dacă numai un număr finit de fibre $f^{-1}(d)$, pentru $d$ din $\mathbb{P}^1$ (fibre pe care le vom nota cu $\widetilde{X}_d$) sunt singulare, iar acelea care sunt au drept singularități numai câte un punct dublu ordinare. Presupun că fac o extindere a corpului astfel încât toate aceste singularități să fie definite de ecuații peste corpul nou.

Toată această construcție depinde de alegerea scufundării și a $L$-ului. Însă există un rezultat ce ne spune că există măcar o scufundare și un $L$ corespunzător astfel încât $f$-ul rezultant să fie pencil Lefschetz (dacă lucram în caracteristică $0$ exista un $L$ în fiece scufundare).

Notăm cu $U$ mulțimea $d$-urilor din $\mathbb{P}^1$ pentru care fibra e netedă, cu $S$ complementara lui $U$ și cu $j$ aplicația de incluziune a lui $U$ în $\mathbb{P}^1$.

Ne vom folosit în continuare de șirul spectral Leray, ale cărui aplicații sunt compatibile cu operatorii Frobenius:
$$E_2^{p,q}=H^p(\mathbb{P}^1,R^q f_* \Qell) \Rightarrow H^{p+q}(\widetilde{X},\Qell)$$
Deci $H^d(\widetilde{X},\Qell)$ admite o filtrare cu spațiile cât $\{E_\infty^{p.q}\}_{p+q=d}$, care la rândul lor sunt subcâturi ale $E_2^{p,q}$-uri, ca urmare e suficient să arăt pentru spațiile din stânga. Dintr-o teoremă de anulare, acestea sunt nenule doar pentru $p$ între $0$ și $2$.

Notez $d-1$ cu $n$ (număr impar). Astfel ne-am redus la a considera următoarele spații:
$$H^2(\mathbb{P}^1,R^{n-1} f_* \Qell);\ H^0(\mathbb{P}^1,R^{n+1} f_* \Qell);\ H^1(\mathbb{P}^1,R^n f_* \Qell)$$
Ne va fi de folos următorul rezultat:

\begin{teo}
(a ciclilor evanescenți) Fie $u$ din $U$. Există $E$ subspațiu vectorial al lui $H^n(\widetilde{X}_u,\Qell)$ (spațiul ciclilor evanescenți) astfel încât:
\begin{enumerate}[I.]
\item Când $E=0$:
\begin{enumerate}[1.]
\item Fasciculul $R^i f_* \Qell$ este constant pentru $i\neq n+1$.
\item Există un șir exact scurt de fascicule pe $\mathbb{P}^1$:
$$0 \rightarrow \bigoplus\limits_{s\in S} (\Qell(-\frac{n+1}{2}))_s \rightarrow R^{n+1}f_*\Qell \rightarrow \underline{H^{n+1}(\widetilde{X}_u,\Qell)} \rightarrow 0$$
\end{enumerate}

\item Când $E\neq 0$ (ceea ce se întâmplă mai frecvent):
\begin{enumerate}[1.]
\item Fasciculul $R^i f_* \Qell$ este constant pentru $i\neq n$.
\item $R^n f_* \Qell = j_* j^* R^n f_* \Qell$.
\item $E$ este subspațiu stabil la acțiunea lui $\pi_1(U,u)$ iar acțiunea pe spațiul cât este trivială.
\item Notând cu $E^\perp$ subspațiul $H^n(\widetilde{X}_u,\Qell)^{\pi_1(U,u)}$, avem că acțiunea pe $\frac{E}{E \cap E^\perp}$ este absolut ireductibilă.
\item Notăm fasciculele constructibile pe U asociate lui $E$ și $E^\perp$ cu $\mathcal{E}$, respectiv $\mathcal{E}^\perp$ (ambele sunt subfascicule ale lui $j^* R^n f_* \Qell$, ce corespunde lui $H^n(\widetilde{X}_u,\Qell)$).
\begin{enumerate}[a)]
\item dacă $E \subseteq E^\perp$, există șirul exact de fascicule pe $\mathbb{P}^1$:
$$0 \rightarrow j_* \mathcal{E}^\perp \rightarrow R^n f_* \Qell \rightarrow j_*(\frac{j^* R^n f_* \Qell}{\mathcal{E}^\perp}) \rightarrow \bigoplus\limits_{s\in S} (\Qell(-\frac{n+1}{2}))_s \rightarrow 0$$
\item dacă $E \nsubseteq E^\perp$, există următoarele două șiruri exacte scurte de fascicule pe $\mathbb{P}^1$:
$$0\rightarrow j_*\mathcal{E} \rightarrow R^n f_* \Qell \rightarrow j_*(\frac{j^* R^n f_* \Qell}{\mathcal{E}}) \rightarrow 0$$
$$0 \rightarrow j_*(\mathcal{E} \cap \mathcal{E}^\perp)  \rightarrow j_*\mathcal{E}  \rightarrow j_*(\frac{\mathcal{E}}{\mathcal{E} \cap \mathcal{E}^\perp})  \rightarrow 0$$
\end{enumerate}
\item Produsul cup pe coomologia lui $\widetilde{X}_u$ induce o biliniară simplectică
$$\psi : \frac{E}{E \cap E^\perp} \times \frac{E}{E \cap E^\perp} \rightarrow \Qell(-n)$$
care este echivariantă relativ la acțiunea lui $\pi_1(U,u)$, iar aplicația canonică rezultantă:
$$\pi_1(U,u) \rightarrow Sp(\frac{E}{E \cap E^\perp},\psi)$$
are imaginea deschisă și densă.
\end{enumerate}
\end{enumerate}
\end{teo}

Mai departe, pot presupune că există $u_0$ punct al lui $U$ definit peste $\mathbb{F}_q$ astfel încât $\widetilde{X}_{u_0}$ admite o secțiune hiperplană netedă $Z_0$, la rândul ei definită peste $\mathbb{F}_q$ (făcând eventual extinderi), de dimensiune, firește, $d-2$ (fiind tot pară, voi putea aplica un raționament prin inducție mai jos).

Demonstrăm mai întâi aserțiunea despre spațiile $H^2(\mathbb{P}^1,R^{n-1} f_* \Qell)$ și $H^0(\mathbb{P}^1,R^{n+1} f_* \Qell)$.

Dat fiind că fasciculul $R^{n-1} f_* \Qell$ este constant în ambele cazuri, el are fibra $H^{n-1}(\widetilde{X}_u,\Qell)$, din teorema de schimbare proprie a bazei. Deci $H^2(\mathbb{P}^1,R^{n-1} f_* \Qell) = H^2(\mathbb{P}^1,\underline{H^{n-1}(\widetilde{X}_u,\Qell)}) = H^2(\mathbb{P}^1,\Qell) \otimes H^{n-1}(\widetilde{X}_u,\Qell) = \Qell(-1)\otimes H^{n-1}(\widetilde{X}_u,\Qell) = H^{n-1}(\widetilde{X}_u,\Qell)(-1)$. Pentru ultimul modul putem aplica ipoteza de inducție, deoarece se scufundă (din teorema Lefschetz slabă) în $H^{n-1}(Z,\Qell)(-1)$, pentru care aplic ipoteza de inducție.

Pentru $H^0(\mathbb{P}^1,R^{n+1} f_* \Qell)$, în cazul în care $E$ e nenul, argumentul funcționează asemănător cu cel precedent, cu deosebirea că aplicăm surjectivitatea morfismului Gysin, care schimbă și rangul coomologiei de la $n+1$ la $n-1$ pentru $Z$ pentru a putea funcționa inducția. Când $E$ este nul, nu fac decât să trec de la șirul exact scurt din teoremă la șirul exact lung în coomologie și să aplic acolo inducția.

Ne-a rămas $H^1(\mathbb{P}^1,R^n f_* \Qell)$. Când $E$ este nul, $R^{n} f_* \Qell$ este constant și cum $H^1(\mathbb{P}^1,\Qell)=0$, am și $H^1(\mathbb{P}^1,R^n f_* \Qell)=0$ și deci nu am nimic de demonstrat.

Când $E$ este nenul, trec prin $j_*$ incluziunea de fascicule de pe $U$ într-una pe $\mathbb{P}^1$:
$$0 \subseteq j_*(\mathcal{E} \cap \mathcal{E}^\perp) \subseteq j_*\mathcal{E} \subseteq R^n f_* \Qell$$
(la ultimul am aplicat punctul 2 din cazul II al teoremei).

Subcazul simplu este $E \subseteq E^\perp$. Notând cu $\mathcal{F}$ conucleul morfismului $j_* \mathcal{E}^\perp \rightarrow R^n f_* \Qell$, ce este izomorf cu nucleul morfismului $j_*(\frac{j^* R^n f_* \Qell}{\mathcal{E}^\perp}) \rightarrow \bigoplus\limits_{s\in S} (\Qell(-\frac{n+1}{2}))_s$, putem sparge șirul exact din teoremă în două șiruri exacte scurte:
$$0 \rightarrow j_* \mathcal{E}^\perp \rightarrow R^n f_* \Qell \rightarrow \mathcal{F} \rightarrow 0$$
$$0 \rightarrow \mathcal{F} \rightarrow j_*(\frac{j^* R^n f_* \Qell}{\mathcal{E}^\perp}) \rightarrow \bigoplus\limits_{s\in S} (\Qell(-\frac{n+1}{2}))_s \rightarrow 0$$
Cum $ j_*(\frac{j^* R^n f_* \Qell}{\mathcal{E}^\perp})$ e constant, are $H^1$ nul (suntem pe $\mathbb{P}^1$), și rezultă că în al doilea șir exact lung corespunzător, morfismul conectant de la $H^0$ la $H^1$ e surjectiv, și rezultă ce trebuie pentru $H^1(\mathbb{P}^1,\mathcal{F})$. La fel, $ j_* \mathcal{E}^\perp$ e constant și deci modulul nostru, $H^1(\mathbb{P}^1,R^n f_* \Qell)$ se scufundă în $H^1(\mathbb{P}^1,\mathcal{F})$ (considerând primul șir exact lung corespunzător). De aici rezultă concluzia.

Trecem la subcazul ce prezintă probleme, $E \nsubseteq E^\perp$. Din cele două șiruri exacte lungi corespunzătoare șirurilor exacte scurte din teoremă, obțin că este suficient să arăt aserțiunea pentru modulul $H^1(\mathbb{P}^1,j_*(\frac{\mathcal{E}}{\mathcal{E} \cap \mathcal{E}^\perp}))$ (am folosit că $j_*(\mathcal{E} \cap \mathcal{E}^\perp)$ și $j_*(\frac{j^* R^n f_* \Qell}{\mathcal{E}})$ sunt constante).

În primul rând, pot arăta doar $|\alpha|\leq q^{\frac{d+1}{2}}$ (utilizând dualitatea și forma simplectică).

În al doilea rând, dat fiind că $supp\ coker(j_!(\frac{\mathcal{E}}{\mathcal{E} \cap \mathcal{E}^\perp}) \rightarrow j_*(\frac{\mathcal{E}}{\mathcal{E} \cap \mathcal{E}^\perp})) = S$ (zero-dimensional și finit), am că $H^1(\mathbb{P}^1,coker)=0$, și ca urmare morfismul indus pe $H^1$ este surjectiv. Am redus la a studia problema pe $H^1(\mathbb{P}^1,j_!(\frac{\mathcal{E}}{\mathcal{E} \cap \mathcal{E}^\perp})) = H^1_c (U, \frac{\mathcal{E}}{\mathcal{E} \cap \mathcal{E}^\perp})$. Vom refolosi notația $\mathcal{F}$ pentru a denota fasciculul $\frac{\mathcal{E}}{\mathcal{E} \cap \mathcal{E}^\perp}$ pe $U$, iar $\mathcal{F}_0$ va fi fasciculul corespunzător pe $U(\mathbb{F}_q)$. Notăm cu $V$ $\pi_1(U_0,u)$-reprezentarea corespunzătoare. Reamintim că avem biliniara simplectică $\pi_1(U,u)$-echivariantă:
$$\psi : V \times V \rightarrow \Qell(-n)$$
iar aplicația canonică rezultantă $\pi_1(U,u) \rightarrow Sp(\frac{E}{E \cap E^\perp},\psi)$ are imaginea deschisă și densă.

Definim, prin analogie cu funcția zeta, $L$-funcția fasciculului $\mathcal{F}_0$ ca:
$$L(U_0,\mathcal{F}_0,t)=\prod\limits_{x \in U} det(1-t^{deg(x)}F^{deg(x)}, \mathcal{F}_x)^{-1}$$
care este egală dintr-o formulă de tip Lefschetz cu
$$\prod\limits_{i=0}^2 det(1-tF_{\mid H^i_c(U,\mathcal{F})})^{(-1)^{i+1}}$$
Însă dat fiind că adică $V$ este reprezentare absolut ireductibilă, $H^2_c$, fiind egal cu dualul spațiului de coinvarianți, este nul. Din faptul că $U$ este afină (dacă ar fi toată dreapta proiectivă, putem aplica tot raționamentul mai întâi scoțând $0$, iar apoi scoțând $\infty$), $H^0_c$ este tot nul. Deci $L$-funcția conține doar termenul corespunzător lui $H^1_c$, care se află în atenția noastră, și este în particular polinomială cu coeficienți raționali (de aici rezultă că valorile proprii sunt algebrice).

Întâi arătăm că este suficient să demonstrăm că pentru orice $x$ și pentru orice $\alpha$ valoare proprie a operatorului Frobenius asociat fibrei $\mathcal{F}_x$ am $|\alpha|=q^{\frac{n}{2} deg(x)}$. Cum:
$$L(U_0,\mathcal{F_0},t)=\prod\limits_{x\in U_0} det(1-F^{deg(x)}t^{deg(x)},\mathcal{F}_x)^{-1} = \prod\limits_{x\in U_0} \prod\limits_{i=1}^r (1-t^{deg(x)}\alpha_{x,i})^{-1}$$
a cărei absolut convergență (din faptul că $|\alpha_{x,i}|=q^{\frac{n}{2} deg(x)}$) este echivalentă cu a seriei:
$$\sum\limits_{x\in U_0} rq^{deg(x)\frac{n}{2}}t^{deg(x)} \leq \sum\limits_{s\geq 1} rq^s q^{s\frac{n}{2}} t^s$$
ultima fiind convergentă pentru $|t|<q^{-\frac{n}{2}-1}$.
Deci în acea zonă nu sunt poli, ca urmare $q^{-\frac{n}{2}-1}<\alpha^{-1}$, ceea ce trebuia demonstrat.

Mai departe vom vedea că putem să arătăm doar {\it condiția de raționalitate pe fibre}, i.e. că pentru orice $x$ din $U$, $det(1-F^{deg(x)}t_{\mid\mathcal{F}_x})$ este polinom cu coeficienți raționali ({\bf lema principală a lui Deligne}).

Pentru aceasta, vom afirma și demonstra în primul rând că pentru orice $k$ natural, $det(1-F^{deg(x)}t^{deg(x)}_{\mid\mathcal{F}^{\otimes 2k}_x})^{-1}$ este serie formală cu coeficienți numere raționale pozitive. Notând $T=t^{deg(x)}$, avem:
\begin{align*}
T\frac{d}{dT}log\ det(1-F^{deg(x)}t^{deg(x)}_{\mid\mathcal{F}^{\otimes 2k}_x})^{-1}&=\sum\limits_{r\geq 1} Tr(F^{r\ deg(x)}_{\mid\mathcal{F}^{\otimes 2k}_x})T^r\\
&= \sum\limits_{r\geq 1} Tr(F^{r\ deg(x)}_{\mid\mathcal{F}_x})^{2k} T^r \in \mathbb{Q}_+[[t]]
\end{align*}
iar inversând operațiile aplicate la început se păstrează caracterul rațional pozitiv.

A doua afirmație pe care o facem este că dacă $\alpha$ este pol pentru $L(U_0,\mathcal{F}^{\otimes 2k},t)$ și toți polii sunt situați pe raza de convergență, iar $\alpha_x$ este zerou pentru $det(1-F^{deg(x)}t^{deg(x)}_{\mid\mathcal{F}^{\otimes 2k}_x})$, primul are modulul mai mic ca al doilea.

Ca să demonstrăm aceasta, notăm $det(1-F^{deg(x)}t^{deg(x)}_{\mid\mathcal{F}^{\otimes 2k}_x})^{-1}$ cu $f_x(t)$, și avem $L(U_0,\mathcal{F}^{\otimes 2k},t)=\prod_x f_x(t)$. Prin urmare, din pozitivitate rezultă că raza de convergență a lui $L$ e mai mică decât a oricărui $f_x$ și deci $\alpha\leq\alpha_x$.

Putem trece acum la demonstrația lemei principale a lui Deligne.

Dat fiind că $U$ este afină, $H^0_c(U,\mathcal{F}^{\otimes 2k})=0$, iar
\begin{align*}
H^2_c(U,\mathcal{F}^{\otimes 2k}) &\simeq (H^0(U,(\mathcal{F}^{\vee})^{\otimes 2k}))^{\vee}(-1)\\
&\simeq ((((\mathcal{F}^\vee)^{\otimes 2k})_u)^{\pi_1(U,u)})^{\vee}(-1)\\
&\simeq ((\mathcal{F}^{\otimes 2k})_u)_{\pi_1(U,u)}(-1)\\
&\simeq ((\mathcal{F}^{\otimes 2k})_u)_{Sp(\mathcal{F}_u,\psi)}(-1)\\
&\simeq \Qell(-kn-1)^N
\end{align*}
pentru un anume $N$, ultimul izomorfism rezultând din teoria reprezentării grupurilor simplectice.

Rezultă că numitorul $L$-funcției este $(1-q^{kn+1}t)^N$, și deci toți polii sunt situați pe raza de convergență, $q^{-kn-1}$. A doua afirmație făcută mai sus ne spune că în acest caz:
$$q^{(-kn-1)deg(x)}\leq|a^{2k}|^{-1}$$
Scoțând radical de ordin $2k$ și ducând pe $k$ la infinit rezultă concluzia.

Rămâne să arătăm condiția de raționalitate pe fibre. Ne folosim de următorul rezultat:

\begin{lema}
Fie $\mathcal{G}_0$ un fascicul pe $U_0$ astfel încât $\mathcal{G}$ e constant. Grupul Galois absolut al lui $\mathbb{F}_q$, $G_{\mathbb{F}_q}$, este cât al lui $\pi_1(U_0,u)$. Atunci aplicația canonică $\pi_1(U_0,u) \rightarrow GL(\mathcal{G}_u)$ factorizează prin $G_{\mathbb{F}_q}$ și deci există $\alpha_1,...,\alpha_k$ unități $\ell$-adice astfel încât pentru orice $x$, $det(1-tF_x, \mathcal{G}_u)=\prod_{i=1}^k (1-\alpha_i^{deg(x)}t)$.
\end{lema}

\begin{proof}
Dat fiind că $\mathcal{G}$ e constant, $\pi_1(U,u)$ acționează trivial, și de aici rezultă factorizarea prin $G_{\mathbb{F}_q}$.
\end{proof}

Avem $\zeta_{\widetilde{X}_x}(t)=\prod_i det(1-tF^{deg(x)}_{\mid H^i(\widetilde{X}_x,\Qell)})^{(-1)^{i+1}}=\prod_i det(1-tF^{deg(x)}_{\mid (R^i f_* \Qell)_x})^{(-1)^{i+1}}$. Și cum toate fasciculele $R^i$ sunt constante în afară de $R^n$, iar acesta din urmă se poate filtra în $\frac{R^n f_* \Qell}{\mathcal{E}}$, $\frac{\mathcal{E}}{\mathcal{E} \cap \mathcal{E}^\perp}$ și $\mathcal{E} \cap \mathcal{E}^\perp$, unde primul și ultimul sunt constante. Deci putem scrie funcția ca:
$$\zeta_{\widetilde{X}_x}(t)=\frac{(1-\alpha_1^{deg(x)}t)...(1-\alpha_r^{deg(x)}t)}{(1-\beta_1^{deg(x)}t)...(1-\beta_s^{deg(x)}t)}det((1-F_xt)_{\mid V})$$
unde membrul stâng este clar rațional. Dacă arătăm că fracția din membrul drept este rațională, va reieși și raționalitatea ultimului termen, pe care dorim să o arătăm.

Este, însă, suficient să arătăm că pentru orice $\tau \in Gal(\frac{\overline{Q_\ell}}{Q})$ există $x \in U_0$ de grad $k$ astfel încât:
\begin{enumerate}
\item nu există $i,j$ cu $(\frac{\alpha_i}{\beta_j})^k=1$
\item nu există $i$ astfel încât $\frac{\alpha_i}{\tau(\alpha_i)}$ să fie rădăcină netrivială a unității (și la fel pentru $\beta$-uri)
\item $\alpha$-urile și $\beta$-urile ridicate la puterea $k$ (precum și imaginile lor prin $\tau^{-1}$) nu sunt în $Spec(F_{x\mid V})$
\end{enumerate}
Astfel, orice asemenea $\tau$ lasă fracția pe loc, ca urmare ea are coeficienții raționali.

Pentru a arăta însă presupunerea făcută, ne vom folosi de:

\begin{prop}
Pentru orice $\lambda$ unitate $\ell$-adică, mulțimea $Z=\{\sigma \in \pi_1(U_0,u) \mid \lambda^{v(\sigma)} \in Spec(\sigma_{\mid V})\}$ este închisă în $\pi_1(U_0,u)$ și de măsură Haar nulă.
\end{prop}

\begin{proof}
Considerăm grupul de asemănări simplectice:
$$GSp(V,\psi)=\{\alpha\in GL(V) \mid \text{ există } \mu \text{ astfel încât pentru orice } x,y \text{ avem } \psi(\alpha(x),\alpha(y))=\mu\psi(x,y)\}$$
Din șirul exact de grupuri de mai devreme, orice acțiune a unui $\sigma$ din $\pi_1(U_0,u)$ este o asemănare cu $\mu(\sigma)=q^{v(\sigma)}$.

Fie mulțimea:
$$H=\{(\alpha,n)\in GSp(V,\psi) \times \hat{\mathbb{Z}} \mid \mu(\alpha)=q^n\}$$
și aplicația naturală $\rho_1$ de la $\pi_1(U_0,u)$ la $H$.

Pentru simplificarea demonstrației, vom presupune că $q$ este rest pătratic modulo $\ell$.

Pot identifica $H$ cu $H'=Sp(V) \times \hat{\mathbb{Z}}$ prin aplicația $(\alpha,n)\mapsto(q^{\frac{n}{2}}\alpha,n)$ și este suficient să arăt concluzia pentru mulțimea $Z'=\{(\alpha,n)\mid\lambda'\in Spec(\alpha)\}$ (unde am notat $\lambda'=q^{\frac{n}{2}}\lambda$).

Clar este o mulțime închisă, fiind dată de ecuații determinantale.

Fixând $n$, notez $P_n$ polinomul minimal al lui $\lambda'^n$ peste $\Qell$. Mulțimea $Z'_n=\{\alpha\in Sp(V,\psi)\mid \lambda'^n \in Spec(\alpha)\}$ este algebrică, deci de măsură Haar nulă. În final, putem zice:
$$\mu(Z')=\int\limits_{n \in \hat{\mathbb{Z}}} \mu(Z'_n) = 0$$
\end{proof}

În sfârșit putem arăta existența punctului. Din teorema de densitate Cebotarev, rezultă că este suficient să arătăm că mulțimea acelor $\sigma$ ce satisfac acele proprietăți 1-3 de mai sus (cu $k=v(\sigma)$, în cazul în care este întreg, și cu $F_{x\mid V}$ înlocuit de $\sigma$) conține un deschis de măsură Haar nenulă.

Însă propoziția anterioară arată că mulțimea elementelor ce satisfac (3) este deschisă de măsură Haar totală, iar mulțimea elementelor eliminate de condițiile (1) și (2) este clar neglijabilă.

S-a demonstrat deci ipoteza lui Riemann.




















\bibliographystyle{plain}
\bibliography{etale}
\end{document}
