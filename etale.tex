\documentclass[13pt]{book}
\usepackage[romanian]{babel}
\usepackage{euler}
\usepackage{xltxtra}
\usepackage{color}
\usepackage{indentfirst}
\usepackage{mathtools}
\usepackage{amsmath}
\usepackage{amsfonts}
\usepackage{amsthm}
\usepackage{amssymb}
\usepackage{listings}
\usepackage[margin=1.1in]{geometry}
\usepackage[linktoc=none]{hyperref}
\usepackage{url}
\usepackage[nottoc]{tocbibind}
\usepackage[raggedright]{titlesec}
\usepackage{young}
\setmainfont{URW Palladio L}
\setcounter{secnumdepth}{-1}
\setcounter{chapter}{0}
\lstset{
language=Haskell,
basicstyle=\ttfamily\fontsize{2.33mm}{4mm}\selectfont,
numbers=left,
numberstyle=\ttfamily\fontsize{2.6mm}{3mm}\selectfont,
stepnumber=3,
numbersep=10pt,
frame=single
}

\begin{document}

\newtheorem{teo}{\bf Teorema}[section]
\newtheorem{cor}[teo]{\bf Corolarul}
\newtheorem{lema}[teo]{\bf Lema}
\newtheorem{prob}[teo]{\bf Problema}
\newtheorem{conj}[teo]{\bf Conjectura}
\newtheorem{propr}[teo]{\bf Proprietatea}
\newtheorem{prop}[teo]{\bf Propoziția}
\newtheorem{alg}[teo]{\bf Algoritmul}
\theoremstyle{remark}
\newtheorem{obs}[teo]{\bf Observația}
\newtheorem{obss}[teo]{\bf Observațiile}
\newtheorem{ex}[teo]{\bf Exemplul}
\newtheorem{exs}[teo]{\bf Exemplele}
\theoremstyle{definition}
\newtheorem{defi}[teo]{\bf Definiția}

\pagenumbering{roman}
\setcounter{page}{1}
\fontsize{3.9mm}{5mm}\selectfont
\pagestyle{empty}
\begin{center}

\LARGE UNIVERSITATEA DIN BUCUREȘTI\\[0.5cm]
\LARGE Facultatea de Matematică și Informatică\\[3cm]

\Large Disertație masterală\\[3.5cm] 
 
\rule{8cm}{0.5mm}\\[0.8cm]
{ \huge \bfseries Étale stuff}\\[0.2cm]
\rule{8cm}{0.5mm}\\[2.5cm]

\begin{minipage}{0.4\textwidth}
\begin{flushleft} \large
\emph{Autor:}\\
Andrei Sipoș
\end{flushleft}
\end{minipage}
\begin{minipage}{0.5\textwidth}
\begin{flushright} \large
\emph{Profesor coordonator:} \\
Lect. dr. Victor Vuletescu
\end{flushright}
\end{minipage}

\vfill

{\large București, 2014}

\end{center}
\newpage
\phantom{X}
\newpage

\pagestyle{headings}
\tableofcontents
\newpage

\setlength{\parskip}{1.5ex plus 0.5ex minus 0.2ex}
\chapter{Introducere}

Acest text își propune să:

\begin{itemize}
\item enunțe cadrul de desfășurare al coomologiei étale
\item prezinte versiuni étale ale unor rezultate fundamentale în topologie, precum dualitatea Poincaré sau formulele de tip Lefschetz
\item aplice aceste rezultate la studiul funcțiilor zeta asociate varietăților peste corpuri finite
\end{itemize}

\setcounter{secnumdepth}{2}

\chapter{Situl étale}
\pagenumbering{arabic}
\setcounter{page}{1}

Pîs pîs pîs

\section{Morfisme étale}

\begin{defi}
Un morfism de inele $A \rightarrow B$ se numește {\bf plat} dacă functorul $B \otimes_A \cdot :  A\textnormal{-Mod} \rightarrow B\textnormal{-Mod}$ este exact.
\end{defi}

\begin{defi}
Un morfism de varietăți (sau scheme) $\phi: Y \rightarrow X$ este {\bf plat} dacă morfismele locale $\mathcal{O}_{X,\phi(y)} \rightarrow \mathcal{O}_{Y,y}$ sunt plate pentru orice $y$ din $Y$.
\end{defi}

\begin{defi}
Un morfism local de inele locale $f: A \rightarrow B$ se numește {\bf neramificat} dacă $A/\mathfrak{m}_A \hookrightarrow B/f(\mathfrak{m}_A)B$ este o extindere finită și separabilă.
\end{defi}

%definiția morfismului de tip finit între varietăți

\begin{defi}
Un morfism de varietăți (sau scheme) $\phi: Y \rightarrow X$ este {\bf neramificat} dacă este de tip finit și morfismele locale $\mathcal{O}_{X,\phi(y)} \rightarrow \mathcal{O}_{Y,y}$ sunt neramificate pentru orice $y$ din $Y$.
\end{defi}

\begin{defi}
Un morfism (regulat) între două varietăți este {\bf étale} dacă este plat și neramificat.
\end{defi}

Morfismele étale au următoarele proprietăți:

\begin{prop}
\begin{enumerate}
\item Orice imersie deschisă este étală.
\item Compunerea a două morfisme étale este étală.
\item Un morfism care este schimbare de bază a unui morfism étale este étale.
\item Dacă $\phi \circ \psi$ și $\phi$ sunt étale, atunci și $\psi$ este étale.
\end{enumerate}
\end{prop}

De acum încolo vom lucra cu o varietate $X$ peste un corp algebric închis $k$.

O vecinătate étală a unui punct $x$ din $X$ este o aplicație étală $\phi: U \rightarrow X$ împreună cu un punct $u \in U$ cu $\phi(u)=x$. Un morfism de vecinătăți étale $(V,v) \rightarrow (U,u)$ este o aplicație regulată de la $V$ la $U$ care duce pe $v$ în $u$ (dacă există, este unică, din anumite proprietăți ale morfismelor étale). Am obținut astfel o categorie index și putem defini {\bf inelul local în $x$ pentru topologia étală} ca fiind:
$$\mathcal{O}_{X, \bar{x}} = \varinjlim_{(U,u)} \Gamma(U,\mathcal{O}_U)$$

Dat fiind că orice vecinătate Zariski, fiind imersie deschisă, este étală, din proprietatea limitei inductive avem un morfism natural
$$\mathcal{O}_{X, x} \rightarrow \mathcal{O}_{X, \bar{x}}$$

\section{}

\chapter{De la étale la $l$-adic}

\chapter{Numărarea punctelor}

Problema pe care urmează să o formulăm a pornit de la cea a numărării punctelor de pe o curbă eliptică. Ne este cunoscută din studiul acelor curbe inegalitatea Hasse-Weil, care spune că pentru orice curbă eliptică $X$ definită peste un corp finit $\mathbb{F}_q$, dacă notăm $N_m(X)=\#X(\mathbb{F}_{q^m})$, are loc relația:
$$|N_m(X) - (q^m+1)| \leq 2\sqrt{q^m}$$

Mai precis, există două numere algebrice $\alpha_1,\alpha_2$ de modul $\sqrt{q}$ astfel încât pentru orice $m$:
$$N_m(X) = 1 - \alpha_1^m - \alpha_2^m + q^m$$

André Weil a propus următoarea generalizare:
\begin{teo}
(Conjecturile Weil) Fie $X$ o varietate proiectivă netedă definită peste $\mathbb{F}_q$ de dimensiune $d$. Atunci:
\begin{enumerate}
\item există $2d$ numere naturale $b_0,...,b_{2d}$ și $\{a_{j,s}\}_{j\in\overline{0,2d}\atop s\in\overline{1,b_j}}$ numere complexe astfel încât pentru orice $m$ am:
$$N_m(X)=\sum\limits_{j=0}^{2d} (-1)^j (\sum\limits_{s=1}^{b_j} \alpha_{j.s}^m)$$
Mai mult, $b_0=b_{2d}=1, \alpha_{0,1}=1, \alpha_{2d,1}=q^d$.
\item pentru orice $j$, $b_j=b_{2d-j}$, iar $(\frac{q^d}{\alpha_{2d-j,1}},...,\frac{q^d}{\alpha_{2d-j,b_j}})$ e o permutare a enumerării $(\alpha_{j,1},...,\alpha_{j,b_j})$.
\item pentru orice $j,s$, $\alpha_{j,s}$ e număr algebric de modul $q^{\frac{j}{2}}$.
\end{enumerate}
\end{teo}

\begin{obs}
Se observă că dacă X este curbă eliptică se reconstituie relația de mai devreme, cu $b_1=2=dim\ H^1(\mathbb{C}/\Lambda, \mathbb{Q})$ (pentru $\Lambda$ o latice în planul complex).
\end{obs}

Un mod mai pragmatic de a exprima conjecturile Weil este reprezentat de instrumentul funcțiilor generatoare.

Ne bazăm pe identitatea formală:
$$log(\frac{1}{1-x})=\sum\limits_{m=1}^{\infty} \frac{x^m}{m}$$
sau
$$\frac{1}{1-x}=exp(\sum\limits_{m=1}^{\infty} \frac{x^m}{m})$$
ce se poate verifica via expansiune în serie Taylor în jurul lui zero.

Definind $Z_X(t) = exp(\sum\limits_{m=1}^{\infty} N_m \frac{t^m}{m})$, obținem din punctul 1 al conjecturilor:
\begin{align*}
Z_X(t)&=exp(\sum\limits_{m=1}^{\infty} \sum\limits_{j=0}^{2d} (-1)^j \sum\limits_{s=1}^{b_j} \alpha_{j,s}^m \frac{t^m}{m})\\
&=\prod\limits_{j=0}^{2d} (\prod_{s=1}^{b_j} exp(\sum\limits_{m=1}^{\infty} \frac{(a_{j,s} t)^m}{m}))^{(-1)^j}\\
&=\prod\limits_{j=0}^{2d} (\frac{1}{\prod\limits_{s=1}^{b_j}(1-\alpha_{j,s}t)})^{(-1)^j}\\
&=\prod\limits_{j=0}^{2d} P_j(t)^{(-1)^{j+1}}
\end{align*}
unde am notat $P_j(t)=\prod\limits_{s=1}^{b_j}(1-\alpha_{j,s}t)$ (și am $P_0(t)=1-t$, $P_{2d}(t)=1-q^dt$).

Vom deriva acum din punctul 2 o relație pe care o va satisface $Z_X(t)$. Aplicăm relația de permutare între enumerări și obținem:
\begin{align*}
P_{2d-j}(t)&=\prod_s(1-\alpha_{2d-j,s}t)=\prod_s(1-\frac{q^d}{a_{j,s}}t)\\
&=(\prod_s \alpha_{j,s})^{-1} \prod_s(\alpha_{j,s}-q^dt)\\
&=(\prod_s \alpha_{j,s})^{-1} (-1)^{b_j} (q^dt)^{b_j} \prod_s (1-\frac{\alpha_{j,s}}{q^dt})\\
&=(\prod_s \alpha_{j,s})^{-1} (-1)^{b_j} (q^dt)^{b_j}P_j(\frac{1}{q^dt})
\end{align*}

Folosim acum atât simetria $b_j$-urilor cât și permutarea enumerărilor:
\begin{align*}
P_j(t)P_{2d-j}(t)&=(q^dt)^{2b_j} (q^d)^{-b_j} P_j(\frac{1}{q^dt}) P_{2d-j}(\frac{1}{q^dt})\\
&=(q^d)^{\frac{b_j+b_{2d-j}}{2}} t^{b_j+b_{2d-j}}P_j(\frac{1}{q^dt}) P_{2d-j}(\frac{1}{q^dt})\\
\end{align*}

Însă $(\prod\limits_s \alpha_{d,s})^2=(q^d)^{b_d}$, deci $\prod\limits_s \alpha_{d,s}=\pm (q^d)^{\frac{b_d}{2}}$.

Deci pentru indicele $d$ am relația:
$$P_d(t)=\pm (-1)^{b_d} (q^dt)^{b_d} (q^d)^{\frac{b_d}{2}} P_d(\frac{1}{q^dt})$$

Și obțin astfel formula pentru funcția $Z_X$:
\begin{align*}
Z_X(t)&=\prod_{j=0}^{2d} P_j(t)^{(-1)^{j+1}}\\
&=\pm \prod_{j=0}^{2d} P_j(\frac{1}{q^dt})^{(-1)^{j+1}} (q^d)^{-\frac{\sum\limits_j (-1)^{j+1} b_j}{2}} t^{-\sum\limits_j (-1)^{j+1} b_j}\\
&=\pm q^{\frac{d\chi(X)}{2}} t^{\chi(X)} Z_X(\frac{1}{q^dt})
\end{align*}
numită {\bf ecuația funcțională} a lui $Z_X$ (unde am notat $\chi(X)=\sum\limits_j (-1)^j b_j$).

O altă reformulare ne este dată de următoarea substituție:
$$\zeta_X(s)=Z_X(q^{-s})$$

Notând pentru un punct închis $x \in X$ cu $\kappa(x)$ corpul rezidual al său, cu $deg(x)$ gradul extinderii $\kappa(x) : \mathbb{F}_q$ și cu
\begin{align*}
N_m(x) = \left\{
     \begin{array}{lr}
       deg(x) & \text{dacă } deg(x)\ |\ m\\
       0 & \text{altfel}
     \end{array}
   \right.
\end{align*}
avem din cele cunoscute de la teoria schemelor:
$$N_m=\sum\limits_x N_m(x)$$

Obținem rescrierile:
\begin{align*}
Z_X(t)&=exp(\sum_{m\geq 1} N_m \frac{t^m}{m}) = exp(\sum_{m\geq 1} \sum\limits_x N_m(x) \frac{t^m}{m})\\
&=exp(\sum_x \sum_{m \geq 1 \atop deg(x) | m} N_m(x) \frac{t^m}{m})\\
&=exp(\sum_x \sum_{n \geq 1} deg(x) \frac{t^{n\cdot deg(x)}}{n \cdot deg(x)})\\
&=exp(\sum_x \sum_{n\geq 1} \frac{(t^{deg(x)})^n}{n})=exp(\sum_x log \frac{1}{1-t^{deg(x)}})\\
&=exp\ log \prod_x \frac{1}{1-t^{deg(x)}} = \prod_x \frac{1}{1-t^{deg(x)}}
\end{align*}
și deci
$$\zeta_X(s)=\prod_x \frac{1}{1-(q^{deg(x)})^{-s}} = \prod_x \frac{1}{1-(\#\kappa(x))^{-s}}$$
Ultima formulă are sens pentru o schemă oarecare, nu neapărat peste un corp finit. De pildă, înlocuind $X$ cu $Spec(\mathbb{Z})$, apare:
$$\zeta_{Spec(\mathbb{Z})}(s)=\prod_{\mathfrak{p} \in Max(\mathbb{Z})} \frac{1}{1-(\#(\frac{\mathbb{Z}}{\mathfrak{p}}))^{-s}} = \prod_{p\text{ prim}} \frac{1}{1-p^{-s}}$$
binecunoscuta funcție zeta a lui Riemann (punctele închise din $Spec(\mathbb{Z})$ sunt precis idealele maximale ale lui $\mathbb{Z}$).

Atenție, însă: funcția $Z_X$ nu are sens decât pentru scheme definite pentru un corp finit!

Este clar că $s\in\mathbb{C}$ este zerou, respectiv pol, pentru $\zeta_X$ dacă și numai dacă $q^{-s}$ va avea aceeași calitate pentru $Z_X$, proprietate ce va avea loc și invers (AICI TREBUIE REFORMULAT!!).
















\bibliographystyle{plain}
\bibliography{etale}
\end{document}
